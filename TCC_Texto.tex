\documentclass[a4paper,12pt,oneside,openany]{book}	
\input{TesePack}

\begin{document}

\frontmatter
\thispagestyle{empty}

\includegraphics[scale=0.7]{Poli.eps}

\begin{center}
\large{Desenvolvimento de Base de Dados para Treinamento de Redes Neurais de Reconhecimento de Voz Através da Geração de Áudios com Resposta
ao Impulso Simuladas por Técnicas de Data Augmentation}\\
   \vspace{2cm}
\large{Bruno Machado Afonso}\\
\end{center}
   \vspace{3cm}
\hspace{7cm}
\hfill \parbox{8.0cm}{Projeto de Graduação apresentado ao Curso de Engenharia Eletrônica e de Computação da Escola Politécnica,
Universidade Federal do Rio de Janeiro, como parte dos requisitos necessários à obtenção do título de Engenheiro.\\}
   \vspace{1cm}
\hfill \parbox{8.0cm}{Orientador: Mariane Rembold Petraglia} \\
   \vspace{1cm}
\begin{center}
Rio de Janeiro

Julho de 2021
\end{center}

\pagebreak


\begin{center}
\large{Desenvolvimento de Base de Dados para Treinamento de Redes Neurais de Reconhecimento de Voz Através da Geração de Áudios com Resposta
ao Impulso Simuladas por Técnicas de Data Augmentation}\\
   \vspace{1cm}
\large{Bruno Machado Afonso}\\
\end{center}
   \vspace{2cm}
PROJETO DE GRADUAÇÃO SUBMETIDO AO CORPO DOCENTE DO CURSO DE ENGENHARIA ELETRÔNICA E DE COMPUTAÇÃO DA ESCOLA POLITÉCNICA DA
UNIVERSIDADE FEDERAL DO RIO DE JANEIRO COMO PARTE DOS REQUISITOS NECESSÁRIOS PARA A OBTENÇÃO DO GRAU DE ENGENHEIRO ELETRÔNICO E DE COMPUTAÇÃO   
   
   \vspace{1cm}
Autor:
      \vspace{0.5cm}
      \begin{flushright}
         \parbox{10cm}{
            \hrulefill

            \vspace{-.375cm}
            \centering{Bruno Machado Afonso}

            \vspace{0.1cm}
         }
      \end{flushright}
      
      
Orientador:
      \vspace{0.5cm}
      \begin{flushright}
         \parbox{10cm}{
            \hrulefill

            \vspace{-.375cm}
            \centering{Profª. Mariane Rembold Petraglia, Ph. D.}

            \vspace{0.1cm}
         }
      \end{flushright}
      
Examinador:
      \vspace{0.5cm}
      \begin{flushright}
         \parbox{10cm}{
            \hrulefill

            \vspace{-.375cm}
            \centering{Prof. José Gabriel Rodríguez Carneiro Gomes, D. Sc.}

            \vspace{0.1cm}
         }
      \end{flushright}
      
Examinador:
      \vspace{0.5cm}
      \begin{flushright}
         \parbox{10cm}{
            \hrulefill

            \vspace{-.375cm}
            \centering{Prof. Julio Cesar Boscher Torres, D. Sc.}

            \vspace{0.1cm}
         }
      \end{flushright}
      
                        
      \vfill
      
      
\begin{center}
Rio de Janeiro

Julho de 2021
\end{center}


\pagebreak            

% Declaracao
\begin{center}
Declaração de Autoria e de Direitos
\end{center}

\vspace{0.4cm}

Eu, \emph{Bruno Machado Afonso} CPF \emph{136.151.347-02}, autor da monografia \emph{Desenvolvimento de Base de Dados para Treinamento de Redes Neurais de Reconhecimento de Voz Através da Geração de Áudios com Resposta
ao Impulso Simuladas por Técnicas de Data Augmentation}, subscrevo para os devidos fins, as seguintes informações:\\
1. O autor declara que o trabalho apresentado na disciplina de Projeto de Graduação da Escola Politécnica da UFRJ é de sua autoria, sendo original em forma e conteúdo.\\
2. Excetuam-se do item 1. eventuais transcrições de texto, figuras, tabelas, conceitos e ideias, que identifiquem claramente a fonte original, explicitando as autorizações obtidas dos respectivos proprietários, quando necessárias.\\
3. O autor permite que a UFRJ, por um prazo indeterminado, efetue em qualquer mídia de divulgação, a publicação do trabalho acadêmico em sua totalidade, ou em parte. Essa autorização não envolve ônus de qualquer natureza à UFRJ, ou aos seus representantes.\\
4. O autor pode, excepcionalmente, encaminhar à Comissão de Projeto de Graduação, a não divulgação do material, por um prazo máximo de 01 (um) ano, improrrogável, a contar da data de defesa, desde que o pedido seja justificado, e solicitado antecipadamente, por escrito, à Congregação da Escola Politécnica.\\
5. O autor declara, ainda, ter a capacidade jurídica para a prática do presente ato, assim como ter conhecimento do teor da presente Declaração, estando ciente das sanções e punições legais, no que tange a cópia parcial, ou total, de obra intelectual, o que se configura como violação do direito autoral previsto no Código Penal Brasileiro no art.184 e art.299, bem como na Lei 9.610.\\
6. O autor é o único responsável pelo conteúdo apresentado nos trabalhos acadêmicos publicados, não cabendo à UFRJ, aos seus representantes,  ou ao(s) orientador(es), qualquer responsabilização/ indenização nesse sentido.\\
7. Por ser verdade, firmo a presente declaração.\\

      \vspace{0.4cm}
      \begin{flushright}
         \parbox{10cm}{
            \hrulefill

            \vspace{-.375cm}
            \centering{Bruno Machado Afonso}

            \vspace{0.1cm}
         }
      \end{flushright}
      
\pagebreak

% Copyright
      \vspace{0.5cm}

UNIVERSIDADE FEDERAL DO RIO DE JANEIRO \\
Escola Politécnica - Departamento de Eletrônica e de Computação \\
Centro de Tecnologia, bloco H, sala H-217, Cidade Universitária \\ 
Rio de Janeiro - RJ      CEP 21949-900\\
\vspace{0.5cm}
\paragraph{}Este exemplar é de propriedade da Universidade Federal do Rio de Janeiro, que poderá incluí-lo em base de dados, armazenar em computador, microfilmar ou adotar qualquer forma de arquivamento.
\paragraph{}É permitida a menção, reprodução parcial ou integral e a transmissão entre bibliotecas deste trabalho, sem modificação de seu texto, em qualquer meio que esteja ou venha a ser fixado, para pesquisa acadêmica, comentários e citações, desde que sem finalidade comercial e que seja feita a referência bibliográfica completa.
\paragraph{}Os conceitos expressos neste trabalho são de responsabilidade do(s) autor(es).

\pagebreak


% Agradecimento
\begin{center}
\textbf{AGRADECIMENTO}
\end{center}
      \vspace{0.5cm}

Agradeço, em primeiro lugar, à Deus que guiou todos os meus passos nessa jornada acadêmica e continua a guiar os caminhos da minha vida.
Sem Ele, não seria capaz de terminar este projeto e não chegaria nos patamares onde estou. A Ele a glória.

Gostaria de agradecer também aos meus pais, em especial à minha mãe, que graças ao seu infinito empenho, dedicação, amor e carinho que me impulsionavam
a cada dia, estou hoje concluindo esta etapa de minha vida.

Não menos importante, agradeço também à minha orientadora, Mariane Rembold Petraglia, que dedicou seu tempo para que eu pudesse me tornar um profissional
ainda melhor. Agradeço pela paciência, atenção, boa vontade e o voto de confiança que foram depositados em mim.

Agradeço também aos meus colegas de faculdade, pois sem eles eu não estaria nem no meu último período da graduação para finalizar o meu projeto.
Em especial, agradeço aos meus colegas Felipe Claudio e Diogo Nocera, ambos por serem os mais presentes tanto na jornada acadêmica quanto para 
o resto da minha vida.

\pagebreak

% Resumo
\begin{center}
\textbf{RESUMO}
\end{center}
      \vspace{0.5cm}

O tema de reconhecimento de voz se torna cada vez mais relevante graças ao seu amplo uso tecnológico na sociedade, desde 
assistentes pessoais em \textit{smartphones}, automação residencial, até autenticação por voz para aplicações de segurança.

Uma das características mais importantes neste tema é a detecção da Resposta ao Impulso de ambientes acústicos (RIR), que representa
o modelo acústico do ambiente. A RIR é usada no processamento de áudio para identificação e reconhecimento de fontes sonoras em campo
distante, que é formada, no caso do tema de reconhecimento de voz, por uma amostra de voz anecoica convoluída com a RIR, acrescida de um ruído. 

Um dos desafios no reconhecimento de voz é a estimação da RIR em um sinal de voz em campo distante.
Além das técnicas tradicionais de processamento de sinais, diversas soluções de \textit{deep learning} foram propostas para a estimação da RIR,
contudo estas acabam sendo limitadas devido à falta de variedade e quantidade de bases de RIRs disponíveis para treinamento de redes neurais.

Neste contexto, o objetivo deste projeto é de desenvolver um algoritmo, usando técnicas de \textit{data augmentation}, que gera amostras de voz
em campo distante (AVCD), construindo assim uma base de dados para uso em treinamentos de soluções de \textit{deep learning}. 
O algoritmo é composto por dois segmentos de \textit{data augmentation}: o primeiro modifica as características de razão direto-reverberante (DRR)
e tempo de reverberação (T60), partindo de RIRs reais, gerando RIRs simuladas (RIRSM); o segundo gera AVCDs, convoluindo amostras de voz anecoicas
com as RIRSMs e adicionando ruídos à voz reverberada. 
Ao final do trabalho, são exibidos exemplos de AVCDs geradas pelo algoritmo proposto, analisando se os dados gerados são válidos para uso 
em treinamento de redes neurais.

\paragraph{}
\noindent Palavras-Chave: Resposta ao Impulso de sala, \textit{data augmentation}, \textit{deep learning}, reconhecimento de voz.

\pagebreak


% Abstract
\begin{center}
\textbf{ABSTRACT}
\end{center}
      \vspace{0.5cm}

Speech recognition is a very relevant topic in the present days due to it's vast technological usage on modern society
from personal assistants on smartphones, residential automated systems to voice authentication for security applications.

One of the most important characteristics on this topic is the Room Impulse Response detection (RIR), which represents the
acoustic model of the room. The RIR is used on signal processing to identify and recognize far-field audio sources,
which for the speech recognition topic, is the anechoic voice sample convolved with the RIR plus noise signal.

One of the challenges when it comes to speech recognition is to estimate the RIR in a far-field voice sample.
Beyond the traditional signal processing algorithms, many deep learning solutions are proposed for the RIR estimation,
however they end up with limited results due to the lack of variety e quantity of RIR databases available for training.

In this context, the main objective of this project is to develop an algorithm using data augmentation techniques that will
generate far-field voice samples, therefore building a database for deep learning training.
The algorithm is composed of two segments: the first modifies the real RIRs characteristics of direct-to-reverberant ratio (DRR) and
the reverberation time (T60) generating simulated RIRs (RIRSM); the second generates far-field voice samples using the previously created 
RIRSMs, anechoic voice samples and noise signals.
At the end of this work, examples of the generated far-field voice samples by the algorithm are shown and they are analysed to see
if they are valid to be used in neural network training.

\paragraph{}
\noindent Key-words: Room Impulse Response, data augmentation, deep learning, voice recognition.

\pagebreak


% Siglas
\begin{center}
\textbf{SIGLAS}
\end{center}
      \vspace{0.5cm}

\paragraph{}AIR - Aachen Impulse Response database
\paragraph{}AVA - Amostra de voz anecoica
\paragraph{}AVCD - Amostra de voz em campo-distante
\paragraph{}AVR - Amostra de voz reverberada
\paragraph{}DA - \textit{Data Augmentation}
\paragraph{}DL - \textit{Deep Learning}
\paragraph{}DRR - Razão Direto-Reverberante
\paragraph{}RIR - Resposta ao Impulso de Ambiente Acústico
\paragraph{}RIRDA - \textit{Data Augmentation} da Resposta ao Impulso de Ambiente Acústico 
\paragraph{}RIRO - Resposta ao Impulso de Ambiente Original
\paragraph{}RIRSM - Resposta ao Impulso de Ambiente Acústico Simulada
\paragraph{}SNR - Razão Sinal-Ruído
\paragraph{}SRF - Sinal de ruído de fundo
\paragraph{}SRP - Sinal de ruído pontual
\paragraph{}T20 - Tempo de Reverberação (queda de 20 DB)
\paragraph{}T30 - Tempo de Reverberação (queda de 30 DB)
\paragraph{}T60 - Tempo de Reverberação (queda de 60 DB)
\paragraph{}UFRJ - Universidade Federal do Rio de Janeiro 
\paragraph{}VA - Voz anecoica
\paragraph{}VR - Voz reverberada


\pagebreak









% Table of Contents
% ---------------------------------------------------------------
\tableofcontents
% ---------------------------------------------------------------
% Lista de figuras
% ---------------------------------------------------------------
%\cleardoublepage
%\addcontentsline{toc}{chapter}{Lista de Figuras}
\listoffigures
% ---------------------------------------------------------------
% Lista de Tabelas
% ---------------------------------------------------------------
%\cleardoublepage
%\addcontentsline{toc}{chapter}{Lista de Tabelas}
\listoftables

\mainmatter
\cleardoublepage
% ---------------------------------------------------------------
% Chapter 1 - Introdução
% ---------------------------------------------------------------
\chapter{Introdução}
\label{cap1}
\section{Tema}

\paragraph{}Falar do que se trata o trabalho usando uma vis�o macrosc�pica (tamanho do texto: 1 ou 2 par�grafos no m�ximo).

\paragraph{}Sobre que grande �rea de conhecimento voc� vai falar?

\paragraph{}Dada esta grande �rea, qual � o subconjunto de conhecimento sobre o qual ser� o seu trabalho?

\paragraph{}Qual o problema a ser resolvido?

õ

\section{Delimita��o}

\paragraph{}Realizar uma delimita��o informando de quem � a demanda, em que local, e em que momento no tempo. Eventualmente, pode ser mais f�cil come�ar pensando por exclus�o, ou seja, para quem n�o serve, onde n�o deve ser aplicado, e em seguida pegar o universo que sobra (tamanho do texto: livre).


\section{Justificativa}

\paragraph{}Apresentar o porqu� do tema ser interessante de ser estudado. Cuidado, n�o � a motiva��o particular. Devem ser apresentadas raz�es para que algu�m deva se interessar no assunto, e n�o quais foram suas raz�es particulares que motivaram voc� a estud�-lo (tamanho do texto: livre).


\section{Objetivos}

\paragraph{}Informar qual � o objetivo geral do trabalho, isto �, aquilo que deve ser atendido e que corresponde ao indicador inequ�voco do sucesso do seu trabalho. Pode acontecer que venha a existir um conjunto de objetivos espec�ficos, que complementam o objetivo geral (tamanho do texto: livre, mas cuidado para n�o fazer uma literatura romanceada, afinal esta se��o trata dos objetivos).


\section{Metodologia}

\paragraph{}Como � a abordagem do assunto. Como foi feita a pesquisa, se vai houve valida��o, etc. Em resumo, voc� de explicar qual foi sua estrat�gia para atender ao objetivo do trabalho (tamanho do texto: livre).


\section{Descri��o}

\paragraph{}No cap�tulo 2 ser� .....

\paragraph{}O cap�tulo 3 apresenta ...

\paragraph{}Os .... s�o apresentados no cap�tulo 4. Nele ser� explicitado ...

\paragraph{}E assim vai at� chegar na conclus�o.


% ---------------------------------------------------------------
% Chapter 2 - Análise de Fontes Sonoras e seus Desafios
% ---------------------------------------------------------------
\chapter{Análise de Fontes Sonoras e seus Desafios}
\label{cap2}
\section{Histórico da Pesquisa de Reconhecimento de Voz}

\section{Desafios do Reconhecimento de Voz em Campo Aberto}

% ---------------------------------------------------------------
% Chapter 3 - Data Augmentation da Resposta ao Impulso do Ambiente
% ---------------------------------------------------------------
\chapter{\textit{Data Augmentation} da Resposta ao Impulso do Ambiente}
\label{cap3}
Este capítulo é dedicado à implementação do primeiro algoritmo de \textit{Data Augmentation}, onde são geradas RIRs simuladas (RIRSM)
a partir de RIRs originais (RIRO), que foram gravadas em ambientes diversos e compõem o banco de dados AIR \cite{AIR_Database}. 
São observados os parâmetros de razão Direto-Reverberante (DRR) e tempo de reverberação (T60), que são
inferidos com base em uma RIRO e que serão manipulados pelo algoritmo para gerar RIRs que vão representar modelos acústicos diferentes.

\begin{figure} [H]
    \centering
    \includegraphics[scale=0.6]{flow-rir-aug.png}
    \caption{Fluxo de procedimentos para gerar a RIRSM.}
    \label{fig:flow-rir-aug}
\end{figure}

Este trabalho é uma implementação dos passos demonstrados no artigo \cite{RIR_Data_Aug}. A Figura \ref{fig:flow-rir-aug} especifica 
o fluxo de procedimentos implantados por este algoritmo, onde a DRR e T60 alvos são os valores escolhidos pelo usuário ou sorteados aleatoriamente,
que determinam as características da RIRSM. 

Antes de explicar os métodos usados, é necessário definir duas funções \cite{RIR_Data_Aug}:

\begin{align} 
    h_e(t) &= 
    \begin{cases} \label{eqn:rir-early}
        h(t), & t_d-t_0 \le t \le t_d+t_0 \\
        0, & \text{caso contrário,}
    \end{cases} \\
    h_l(t) &= 
    \begin{cases} \label{eqn:rir-late}
        h(t), & t < t_d - t_0 \\
        h(t), & t > t_d + t_0 \\
        0, & \text{caso contrário,}
    \end{cases}
\end{align}

\noindent
onde $t$ representa o tempo discreto, $t_d$ o tempo que as ondas sonoras diretas, ou seja, sem reflexão, levam da fonte até o destino de gravação,
$t_0$ a janela de tolerância, neste caso definida com o valor 2,5 ms \cite{RIR_Data_Aug}, 
$h(t)$ uma RIR, $h_e(t)$ as primeiras reflexões e $h_l(t)$ as reflexões tardias.
Neste algoritmo, $t_d$ é determinado da forma:

\begin{align} \label{eqn:t_d}
    %t_d = t_{max},\ onde \ \ h(t_{max}) = max(h(t))
    \begin{cases}
        t_d = t_{max},\\
        t_{max}, \ \text{onde} \ h(t_{max}) = max(|h(t)|)
    \end{cases}
    .
\end{align}


\section{Razão Direto-Reverberante (DRR)}


A DRR representa a razão entre a energia sonora da resposta ao impulso que atinge o alvo diretamente e a energia reverberante,
ou seja, que é refletida pelas paredes do ambiente fechado. Este parâmetro é calculado pela seguinte expressão:

\begin{equation} \label{eqn:DRR_def}
    DRR_{dB} = 10 \log_{10} \left( \frac{\sum_t h^2_e(t)}{\sum_t h^2_l(t)} \right).
\end{equation}

Para obter a $DRR_{dB}$ alvo desejada, aplica-se um fator de ganho escalar $\alpha$ na função das primeiras reflexões $h_e(t)$.
De acordo com \cite{RIR_Data_Aug}, para evitar descontinuidades durante o cálculo do fator $\alpha$, reescreve-se 
$h_e(t)$ em duas parcelas, uma que representa a janela direta no pico de intensidade de $h(t)$ e outra
que representa uma janela de resíduo de $h_e(t)$ formando, assim,

\begin{equation} \label{eqn:new-h_e}
    h'_e(t) = \alpha w_d(t) h_e(t) + [1 - w_d(t)]h_e(t),
\end{equation}

\noindent
onde $w_d(t)$ representa uma janela de Hann de duração de 5 ms, considerando-se uma janela de tolerância $t_0 = 2,5$ ms.
Substituindo na Equação (\ref{eqn:DRR_def}) $h_e(t)$ por $h'_e(t)$ e combinando com a Equação (\ref{eqn:new-h_e}), obtém-se
a seguinte equação quadrática;

\begin{equation} \label{eqn:DRR_quad_eqn}
    \begin{aligned} 
        \alpha^2 \sum_t w^2_d(t) h^2_e(t) +
        2 \alpha \sum_t [1 - w_d(t)] w_d(t) h^2_e(t) + \\
        \sum_t [1 - w_d(t)]^2 h^2_e(t) -
        10^{DRR_{dB}/10} \sum_t h^2_l(t)
        = 0 ,
    \end{aligned}
\end{equation}

O parâmetro $\alpha$ desejado será a raiz de maior valor. 
Uma ressalva deste procedimento é que se deve atentar para não escolher uma $DRR_{dB}$ que não seja muito menor que a original,
pois após a transformação de $h_e(t)$ para $h'_e(t)$, dependendo do valor de $\alpha$, é possível incidir em um caso onde
$max(h'_e(t)) < max(h_l(t))$, tornando a RIRSM impraticável.

A Figura \ref{fig:ir-early} exibe um exemplo de sinal $h(t)$ com o seu $h_e(t)$ após ser feita a modificação do DRR.
Comparando $h_e(t)$ com $h'_e(t)$, de acordo com a Figura \ref{fig:drr-da}, nota-se que $h'_e(t)$  está com uma intensidade 
maior, o que condiz a modificação do DRR de -4,5 dB para 4 dB neste exemplo.

\begin{figure}[H]
    \centering
    \includegraphics[scale=0.3]{ir-early.png}
    \caption{Um exemplo de $h(t)$ com $h_e(t)$, onde é feita a modificação do DRR de -4,5 para 4, marcado em vermelho.}
    \label{fig:ir-early}
\end{figure} 

\begin{figure}[H]
    \centering
    \includegraphics[scale=0.3]{drr-da.png}
    \caption{Um exemplo de $h_e(t)$ antes e depois da aplicação do algoritmo, original em azul ($DRR=-4,5$ dB) e o modificado em vermelho ($DRR=4$ dB).}
    \label{fig:drr-da}
\end{figure} 


\section{Tempo de Reverberação (T60)}


O T60, definido na equação \ref{eqn:T60-def}, representa a duração de tempo que leva para a energia sonora da RIR
no alvo decair 60 dB comparado à sua intensidade máxima. Geralmente, devido à dificuldade de se medir uma queda de 60 dB,
o parâmetro medido é o T20 ou o T30 e depois multiplica-se os seus valores por 3 e 2, respectivamente, para obter o T60,
assumindo-se um decaimento exponencial da envoltória da RIR.

\begin{align} \label{eqn:T60-def}
    \begin{cases}
        \text{T60} = t_f-t_i, \ \text{onde}\\
        t_i \rightarrow h(t_i) = max(h(t)) \\
        t_f \rightarrow 10 \log_{10} \left( h^2(t_i) - h^2(t_f) \right) = 60 \text{dB} \\
        %t_i, \ \text{onde} \ h(t_i) = max(h(t)) \\
        %t_f, \ \text{onde} \ 10 \log_{10} \left( h^2(t_i) - h^2(t_f) \right) = 60 \text{dB} \\
        %\text{T60} = t_f-t_i.
    \end{cases}
    .
\end{align}

Para realizar modificações na RIR, é necessário modelar a função das reflexões tardias. De acordo com \cite{RIR_Data_Aug},
um modelo normalmente usado é de um ruído gaussiano exponencialmente decadente, acrescido de um ruído residual, ou seja,

\begin{equation} \label{eqn:h_l-gauss}
    h_m(t) = A e^{-(t - t_o)/ \tau} n(t) u(t - t_o) + \sigma n(t)
    ,
\end{equation}

\noindent
onde $A$ representa o ganho da resposta ao impulso, $\tau$ a taxa de decaimento, $\sigma_m$ o desvio padrão do ruído residual, 
$n(t)$ um ruído gaussiano padrão (média nula e desvio padrão unitário), $t_o$ o valor temporal onde $h_l(t)$
tem o seu primeiro valor não nulo e $u(t)$ um degrau unitário.
Neste trabalho, diferente da implementação do algoritmo em \cite{RIR_Data_Aug}, é considerada apenas a taxa de decaimento
do espectro de frequência por completo da RIR, ao invés de dividi-la em subbandas e analisar 
a taxa de decaimento para cada faixa de frequência.

Os parâmetros $A$, $\tau$ e $\sigma$ são estimados de acordo com o padrão definido na ISO 3382-1 \cite{ISO-3382}.
Seja $T60_d$ o valor de T60 alvo para DA, é possível inferir a taxa de decaimento através da equação

\begin{equation} \label{eqn:decay-rate-t60}
    T60_d = \ln(1000) \tau_d T_s
    ,
\end{equation}

\noindent
onde $\tau_d$ representa a taxa de decaimento alvo e $T_s$ o intervalo de amostragem.
A DA do tempo de reverberação é feita multiplicando-se $h_l(t)$ pela exponencial

\begin{equation} \label{eqn:DA-T60}
    h'_l(t) = h_l(t) e^{-(t - t_o) \frac{\tau - \tau_d}{ \tau \tau_d} }
    .
\end{equation}

Por fim, a RIRSM completa, $h'(t)$, pode ser representada pela equação

\begin{equation} \label{eqn:RIRSM}
    h'(t) = h'_e(t) + h'_l(t)
    .
\end{equation}

A Figura \ref{fig:ir-late} exibe um exemplo de sinal $h(t)$ com o seu $h_l(t)$ após ser feita a modificação do T60.
Comparando $h_l(t)$ com $h'_l(t)$, de acordo com a Figura \ref{fig:t60-da}, nota-se que $h'_l(t)$  está com uma intensidade 
maior, o que condiz a modificação do T60 de 1,38 para 2,6 segundos neste exemplo.

\begin{figure}[H]
    \centering
    \includegraphics[scale=0.3]{ir-late.png}
    \caption{Um exemplo de $h(t)$ com $h_l(t)$, onde é feita a modificação do T60 de 1,38 para 2,6 segundos, marcado em vermelho.}
    \label{fig:ir-late}
\end{figure} 

\begin{figure}[H]
    \centering
    \includegraphics[scale=0.3]{t60-da.png}
    \caption{Um exemplo de um trecho ao final de $h_l(t)$ antes e depois da aplicação do algoritmo, original em azul ($T60=1,38$ s) e o modificado em vermelho ($T60=2,6$ s).}
    \label{fig:t60-da}
\end{figure} 

% ---------------------------------------------------------------
% Chapter 4 - Desenvolvimento de Sinais de Voz Reverberadas Simuladas com Ruídos
% ---------------------------------------------------------------
\chapter{Desenvolvimento de Sinais de Voz Reverberadas Simuladas com Ruídos}
\label{cap4}
Este capítulo é dedicado ao desenvolvimento do segundo algoritmo de \textit{Data Augmentation}, onde são geradas as amostras de voz 
reverberadas em campo-distante (AVCDs) usando: amostras de voz anecóicas (AVAs), RIRSM, sinais de ruído pontuais (SRPs) e de fundo (SRFs).

\begin{figure} [H]
    \centering
    \includegraphics[scale=0.65]{flow-voz-aug.png}
    \caption{Fluxo de procedimentos para gerar a AVCD.} 
    \label{fig:flow-voz-rev}
\end{figure}

Este trabalho é uma implementação do método de DA proposto no artigo \cite{Speech_Rec}. A Figura \ref{fig:flow-voz-rev} especifica 
o fluxo de procedimentos implantados por este algoritmo, onde o SRP, SRF e SNR alvos são aleatoriamente escolhidos, 
dentro de uma base de dados de ruídos e uma faixa de valores definidas pelo usuário, que determinam as características da AVCD. 

\section{Simulação de fala em campo distante} 

Sinais de voz em campo-distante são tipicamente compostos por uma combinação de VR, SRP (assumindo que a fonte do ruído pontual encontra-se 
no mesmo ambiente da VR) e SRF (assumindo que o ruído de fundo não é afetado pelo modelo acústico do ambiente).
É possível modelar uma AVCD conforme a equação

\begin{equation} \label{eqn:AVCD-model}
    S_{cd}[t] = S_a[t] \ast h[t] + \sum_i n_{pi}[t] \ast h[t] + n_f[t]
    ,
\end{equation}

\noindent
onde $S_{cd}[t]$ representa a AVCD, $S_a[t]$ a AVA, $h[t]$ a RIR, $n_{pi}[t]$ o i-ésimo SRP e $n_f[t]$ o SRF.
Neste trabalho, diferente da implementação do algoritmo em \cite{Speech_Rec}, é considerado apenas uma única RIR
para gerar a AVCD, ou seja, os ruídos pontuais são convoluídos com a mesma RIR que é usada para a fonte de voz.

No Algoritmo \ref{alg:AVCD-gen} descreve-se o procedimento que é usado para gerar sinais de voz em campo-distante simulados. 
\bigbreak
\bigbreak

\begin{algorithm} [H] 
    \caption{Procedimentos para gerar AVCD}
    \label{alg:AVCD-gen}

    \KwIn{$fl_p$ : Flag de inclusão de ruído pontual} 
    \KwIn{$fl_g$ : Flag de inclusão de ruído de fundo} 
    \KwIn{$m$ : Quantidade de ruídos pontuais} 
    \KwIn{$SNR_{up}$ : Limite superior de SNR} 
    \KwIn{$SNR_{dw}$ : Limite inferior de SNR} 

    $S_r[t] \gets S_a[t] \ast h[t]$ : Convolução da RIR com AVA

    \If{$fl_p = true$}
    {
        \For{$i = 1$ até $m$}
        {
            Escolha aleatória de um ruído pontual $n_{pi}[t]$ da biblioteca de ruído. \\
            Escolha aleatória de uma SNR Alvo $SNR_t$ compreendida dentro do intervalo $[SNR_{dw},SNR_{up}]$. \\
            Dedução do fator $\alpha$ a partir da $SNR_t$ para corrigir a intensidade de $n_{pi}[t]$. \\
            Escolha aleatória de offset $o_t$ compreendida dentro do intervalo $(0,\text{duração(t)})$. \\
            $S_r[t] \gets S_r[t] + \alpha \text{ offset}(n_{pi}[t] \ast h[t], o_t)$ : Adição de SRP na AVR.
        }
    }
    \If{$fl_g = true$}
    {
        Escolha aleatória de um ruído de fundo $n_f[t]$ da biblioteca de ruído. \\
        Escolha aleatória de uma SNR Alvo $SNR_t$ compreendida dentro do intervalo $[SNR_{dw},SNR_{up}]$. \\
        Dedução do fator $\alpha$ a partir da $SNR_t$ para corrigir a intensidade de $n_f[t]$. \\
        Estender ou encurtar $n_f[t]$ até que $\text{duração }(n_f[t]) = \text{duração}(S_r[t])$
        $S_r[t] \gets S_r[t] + \alpha n_f[t]$ : Adição de SRF na AVR.
    }

\end{algorithm}
%\bigbreak
%\bigbreak
\pagebreak


Neste trabalho, o algoritmo de geração de AVCD usa as RIRSMs geradas através do primeiro algoritmo, diferente do que foi implantado
em \cite{Speech_Rec}, onde foram geradas RIRs de forma completamente digital \cite{RIR_sim_image}, ou seja, sem usar RIRs reais 
como base para \textit{Data Augmentation}.
Nota-se também que o algoritmo permite habilitar ou não o uso de cada tipo de ruído para que possa aumentar a variedade de dados gerados, além
de acomodar mais propósitos de treinamentos de \textit{Deep Learning}.

A Figura \ref{fig:voice-sample} exibe uma amostra de voz anecóica que foi usada para gerar os próximos exemplos de aplicação
do algoritmo. É feita uma comparação entre a convolução da AVA com a RIRO e com a RIRSM, representadas nas Figuras \ref{fig:voice-aug-riro} e 
\ref{fig:voice-aug-rirsm} respectivamente; pode-se notar que a convolução com a RIRSM causa menos modificações no formato do sinal de voz original 
comparado à RIRO, uma vez que a primeira DA foi configurada para simular um ambiente menos reverberante (partindo dos parâmetros de $DRR=-4$ e $T60=1,38$ s
na RIRO para $DRR=10$ e $T60=0,50$ s na RIRSM).

\begin{figure} [H]
    \centering
    \includegraphics[scale=0.3]{voice-sample.png}
    \caption{Exemplo de amostra de voz anecóica.}
    \label{fig:voice-sample}
\end{figure} 

\begin{figure} [H]
    \centering
    \includegraphics[scale=0.3]{voice-aug-riro.png}
    \caption{Exemplo de amostra de voz reverberante, convoluída com uma RIRO, onde $DRR = -4$ e $T60=1,38$ s.}
    \label{fig:voice-aug-riro}
\end{figure} 

\begin{figure} [H]
    \centering
    \includegraphics[scale=0.3]{voice-aug-rirsm.png}
    \caption{Exemplo de amostra de voz reverberante, convoluída com uma RIRSM, onde $DRR = 10$ e $T60=0,50$ s.}
    \label{fig:voice-aug-rirsm}
\end{figure} 

A partir da RIRSM usada nesta aplicação, foram geradas dois exemplos de AVCDs, representadas nas Figuras \ref{fig:voice-noise-14snr} e 
\ref{fig:voice-noise-4snr}, esta com $SNR = 4$ e aquela com $SNR = 14$. Conforme o esperado, observa-se que a AVCD com o menor SNR
possui ruídos claramente mais acentuados comparado à AVCD com o maior SNR.

\begin{figure} [H]
    \centering
    \includegraphics[scale=0.3]{voice-noise-14snr.png}
    \caption{Exemplo de amostra de voz em campo distante cujo $SNR = 14$, representado pela voz reverberada mais os ruídos adicionados pelo segundo método de DA.}
    \label{fig:voice-noise-14snr}
\end{figure} 

\begin{figure} [H]
    \centering
    \includegraphics[scale=0.3]{voice-noise-4snr.png}
    \caption{Exemplo de amostra de voz em campo distante cujo $SNR = 4$, representado pela voz reverberada mais os ruídos adicionados pelo segundo método de DA.}
    \label{fig:voice-noise-4snr}
\end{figure} 

% ---------------------------------------------------------------
% Chapter 5 - Bases de Dados
% ---------------------------------------------------------------
\chapter{Bases de Dados}
\label{cap5}
Este capítulo apresenta as bases de dados que serão usadas para gerar os resultados experimentais.
Para a aplicação da metodologia deste trabalho, é necessário três fontes de dados:

\begin{itemize}
    \item base de dados com amostras de voz anecóicas para convolução com as RIRSMs;
    \item base de dados com RIRs reais para a primeira técnica de aumento de dados, gerando as RIRSMs;
    \item base de dados com SRPs e SRFs para a segunda técnica de aumento de dados, gerando as AVCDs.
\end{itemize}


\section{Base de amostras de voz anecóicas}

A base de AVAs usada consiste na leitura de textos em inglês por 4 pessoas diferentes (duas vozes masculinas e duas femininas)
gravadas em uma câmara anecóica para avaliação de algoritmos de cancelamento de eco acústico. 
O eco foi adicionado artificialmente, de modo a efetuar experimentos controlados e com diferentes tempos de reverberação.
Os arquivos de áudio são disponibilizados em formato WAV, com frequência de amostragem de 16 KHz e cada gravação tem duração
em torno de 5 a 6 segundos. No caso deste trabalho, foram concatenadas duas frases por pessoa na mesma 
gravação devido ao tempo de duração dos ruídos pontuais, que serão adicionados para a geração de AVCDs.

A Tabela \ref{tbl:voice} descreve as gravações usadas neste projeto.

\begin{table} [H]
    \centering
    \caption{Descrição dos textos pronunciados por locutor.}
    \label{tbl:voice}
    \begin{tabularx}{\textwidth}{l|c|p{9cm}} 
        
        \multicolumn{1}{c}{\textbf{Nome}} & \multicolumn{1}{|c}{\textbf{Código}} & \multicolumn{1}{|c}{\textbf{Texto}} \\
        %\textbf{Nome} & \textbf{Código} & \textbf{Texto} \\
        \hline 

        Homen 1 - Texto 1 & H1-T1 & \textit{This food is too spicy he complained. Young man can be very arrogant and rude.} \\
        Homen 1 - Texto 2 & H1-T2 & \textit{So Marcus owned a big shipping company. Their eyes met across the table.} \\
        Homen 2 - Texto 1 & H2-T1 & \textit{Time is running out for the scientists. If you knew Julie like I know Julie.} \\
        Homen 2 - Texto 2 & H2-T2 & \textit{Your new dress is breathtaking darling. Her first book was published last year.} \\
        Mulher 1 - Texto 1 & M1-T1 & \textit{Among them are canvases from a young artist. Building from the ground up is very costly.} \\
        Mulher 1 - Texto 2 & M1-T2 & \textit{Next year we will see several more exibitions. The number of works on view will increase.} \\
        Mulher 2 - Texto 1 & M2-T1 & \textit{An enourmous quake rocked the island. Eventually he hopes to solve all the problems.} \\
        Mulher 2 - Texto 2 & M2-T2 & \textit{Eventually he hopes to solve all the problems. Faulty installation can be blamed for this.} \\
        
    \end{tabularx}
\end{table}

\section{Base de RIRs - Aachen Impulse Response database (AIR)}

A base de AIR \cite{AIR_Database} é um conjunto de respostas ao impulso sonoras que foram medidas em diversas salas
por pesquisadores do Instituto de Acústica (ITA) da Universidade RWTH, Aachen, Alemanha.
O objetivo dessa base é fornecer dados para estudos de algoritmos de processamento de sinais para ambientes reverberantes.

Ela é composta, primariamente, por RIRs binaurais medidas com uma cabeça de manequim em locais com diferentes
propriedades acústicas. É importante frisar também que a base possui gravações com diferentes distâncias entre a fonte sonora
e os microfones para a mesma sala, gerando funções com diferentes valores de DRR.
A base também possui gravações em diferentes direções com o objetivo de auxiliar algoritmos de detecção
de direção da fonte sonora. Para o escopo deste projeto, tais RIRs não serão usadas. 

\begin{table} [H]
    \centering
    \caption{Configurações de RIRs disponíveis na AIR.}
    \label{tbl:rir}
    \begin{tabularx}{\textwidth}{l|c|c|c|l}
        
        \multicolumn{1}{c|}{\textbf{Sala}} & \multicolumn{1}{c|}{\textbf{Descrição}} & \multicolumn{1}{c|}{\textbf{Canais}} &
        \multicolumn{1}{|c|}{\textbf{Cabeça}} & \multicolumn{1}{c}{\textbf{Distâncias (m)}} \\
        %\textbf{Sala} & \textbf{Descrição} & \textbf{Canais} & \textbf{Cabeça} & \textbf{Distâncias (m)} \\
        \hline 

        Booth & cabine de estúdio & E/D & S/N & 0,5/1/1,5 \\
        Office & escritório comercial & E/D & S/N & 1/2/3 \\
        Meeting & sala de reuniões & E/D & S/N & 1,45/1,7/1,9/2,25/2,8 \\
        Lecture & sala de aula & E/D & S/N & 2,25/4/5,56/7,1/8,68/10,2 \\
        Stairway & escadaria aberta & E/D & S/N & 1/2/3 \\
        Aula Carolina & igreja de área 570m² & E/D & S/N & 1/2/3/5/15/20 

    \end{tabularx}
\end{table}

Os ambientes em que foram feitas as gravações de RIRs e suas respectivas configurações são definidos na Tabela \ref{tbl:rir}.
Todos os ambientes usados possuem gravações com ambos os canais esquerdo e direito (E/D), com configuração com ou sem a cabeça
falsa (S/N) e para diferentes distâncias (em metros) entre a fonte que gera o impulso sonoro e o microfone.
As RIRs foram salvas como vetores binários de precisão dupla de ponto flutuante (formato MAT, que pode ser importado
via MATLAB\textregistered).

\section{Base de ruídos - MUSAN}

A base de MUSAN (\textit{A Music, Speech, and Noise Corpus}) \cite{noiseLib} consiste em um conjunto de músicas de diversos gêneros,
amostras de voz de doze línguas e uma variedade de ruídos técnicos, por exemplo tons DTMF e sons de uma máquina de fax, e ruídos de ambiente,
por exemplo sons de animais e chuva.
Ela foi criada primariamente para auxiliar no treinamento de modelos voltados para detecção de atividade de voz. 
Contudo, ela é usada também para teste de algoritmos processamento de sinais na área de áudio, por exemplo, de reconhecimento de voz e orador.
Uma das vantagens dessa base é o fato dela ser uma compilação de áudios com fontes em domínios públicos, facilitando a 
distribuição dos áudios para uso da comunidade científica.

No escopo deste projeto, será usada somente a seção de ruídos da base, que contém seis horas de áudio no total.
A seção de ruídos é composta por sons de curta duração, que são usados como SRPs no segundo processo de 
aumento de dados, e por sons de ambiente, usados como SRFs no mesmo processo.

\begin{table} [H]
    \centering
    \caption{Descrição dos tipos de ruídos pontuais usados da base MUSAN.}
    \label{tbl:noise}
    \begin{tabular}{c|l}

        \multicolumn{1}{c|}{\textbf{Código}} & \multicolumn{1}{c}{\textbf{Descrição}} \\
        %\textbf{Código} & \textbf{Descrição} \\
        \hline 

        RP-1 & miado de gato \\
        RP-2 & madeira sendo lixada \\
        RP-3 & buzina de automóvel \\
        RP-4 & porta abrindo \\
        RP-5 & grampeador \\
        RP-6 & teclado de forno de microondas \\
        RP-7 & \textit{zipper} sendo fechado \\
        RP-8 & latido de cão \\
        RP-9 & batendo em uma porta \\
        RP-10 & espirro \\
        RP-11 & campainha \\
        RP-12 & vibrador de celular \\

    \end{tabular}
\end{table}

\begin{table} [H]
    \centering
    \caption{Descrição dos tipos de ruídos de fundo usados da base MUSAN.}
    \label{tbl:noise-bg}
    \begin{tabular}{c|l}

        \multicolumn{1}{c|}{\textbf{Código}} & \multicolumn{1}{c}{\textbf{Descrição}} \\
        %\textbf{Código} & \textbf{Descrição} \\
        \hline 

        RF-1 & avião decolando em aeroporto \\
        RF-2 & sala de máquinas \\
        RF-3 & estática \\
        RF-4 & sons de floresta \\

    \end{tabular}
\end{table}

As Tabelas \ref{tbl:noise} e \ref{tbl:noise-bg} indicam os ruídos separados da base para gerar AVCDs.
Os arquivos de áudio são disponibilizados em formato WAV, com frequência de amostragem de 16 KHz. 

% ---------------------------------------------------------------
% Chapter 6 - Resultados Experimentais
% ---------------------------------------------------------------
\chapter{Resultados Experimentais}
\label{cap6}
Este capítulo descreve os procedimentos e as configurações feitas para a implementação empírica da metodologia deste trabalho.
Cada seção exibe um conjunto de gráficos relativos ao procedimento implementado para um único exemplo. Os gráficos obtidos para todos
os exemplos se encontram no Apêndice \ref{ApendiceA}.
O código fonte da implementação dos algoritmos propostos se encontra no Apêndice \ref{ApendiceB}.

\section{Configuração dos parâmetros}

As RIRSMs e AVCDs geradas nestes resultados usam as bases de dados de RIR, amostras de voz anecóicas e ruídos descritas no Capítulo \ref{cap5}.
As faixas de DRR, T60 e SNR usados são exibidos na Tabela \ref{tbl:config-param}.

\begin{table} [H]
    \centering
    \caption{Faixas dos parâmetros.}
    \label{tbl:config-param}
    \begin{tabular}{c|p{9cm}}

        \multicolumn{1}{c|}{\textbf{Parâmetro}} & \multicolumn{1}{c}{\textbf{Faixa}} \\
        \hline 

        $DRR_{alvo}$ (dB) & $-6 \le DRR_{alvo} \le 18 $ \\
        $T60_{alvo}$ (s) & $T60_{org} - 1  \le T60_{alvo} \le T60_{org} + 1$ , onde o limite inferior de $T60_{alvo} = 0.2$ \\
        $SNR_{alvo}$ & $3 \le SNR_{alvo} \le 20 $ \\

    \end{tabular}
\end{table}

O $DRR_{alvo}$ segue os valores propostos no artigo \cite{RIR_Data_Aug}, já os valores de $T60_{alvo}$, devido à maior faixa de valores 
de $T60$ das RIRs base de AIR, são limitados a $\pm 1 $ segundo comparado ao valor de $T60_{org}$ da RIR original usada como base para o algoritmo
de DA implementado.

\section{Resultados}

\subsection{\textit{Data Augmentation} do DRR}

Esta seção apresenta os resultados da \textit{Data Augmentation} do DRR. Para isso, foram gerados três exemplos de RIRSMs. Suas configurações
são exibidas na Tabela \ref{tbl:da-drr}, onde $DRR_{org}$ representa o DRR da RIR original, $DRR_{alvo}$ o valor de DRR desejado pelo usuário,
$DRR_{res}$ o valor de DRR resultante após DA e $\rho_{DRR}$ é o erro relativo definido da forma $\rho_{DRR} = abs(DRR_{res} - DRR_{alvo})/DRR_{alvo}$.

\begin{table} [H]
    \centering
    \caption{Exemplos de DA de DRR gerados.}
    \label{tbl:da-drr}
    \begin{tabular}{c|c|c|c}

        \textbf{Exemplo} & 
        \textbf{Sala RIR} & 
        \textbf{Distância (m)} &
        \textbf{Amostra de Voz} \\
        \hline 

        D1 & lecture & 7.1 & H2-T2 \\
        D2 & booth & 1 & H2-T1 \\
        D3 & office & 2 & M2-T2 \\

    \end{tabular}
    \bigbreak
    \bigbreak
    \begin{tabular}{c|c|c|c|c}

        \textbf{Exemplo} & 
        \textbf{$DRR_{org}$ (dB)} & 
        \textbf{$DRR_{alvo}$ (dB)} &
        \textbf{$DRR_{res}$ (dB)} & 
        \textbf{$\rho_{DRR}$ (\%)} \\
        \hline 

        D1 & -4,5 & 10 & 10 & 0 \\
        D2 & 4,7 & -2 & -2 & 0 \\
        D3 & 0,5 & 18 & 18 & 0 \\

    \end{tabular}
\end{table}


Nos exemplos gerados, na faixa determinada para o $DRR_{alvo}$, não houve diferença entre o $DRR_{alvo}$ e o $DRR_{res}$.
É possível observar na Figura \ref{fig:rir-aug-d1} que ocorreu um aumento na seção equivalente a $h_e(t)$ comparado à da Figura \ref{fig:rir-og-d1}
para o exemplo D1.

\begin{figure} [H]
    \centering
    \includegraphics[scale=0.25]{rir-og-d1.png}
    \caption{RIR original do exemplo D1.}
    \label{fig:rir-og-d1}
\end{figure} 

\begin{figure} [H]
    \centering
    \includegraphics[scale=0.25]{rir-aug-d1.png}
    \caption{RIR simulada do exemplo D1.}
    \label{fig:rir-aug-d1}
\end{figure} 

Foi realizado também um experimento subjetivo com uma pessoa sem problemas auditivos, cuja idade é de 25 anos na data do experimento,  
que não possui prévio conhecimento dos resultados gerados.
A pessoa foi colocada em um quarto silencioso e foi usado um fone de ouvido para ouvir os aúdios gerados para este experimento.
O objetivo é avaliar a percepção de “distância” do som, ou seja, a sensação subjetiva do locutor estar falando próximo ou distante
do microfone.
Foram geradas amostras de voz reverberadas (AVR) usando a RIRO e RIRSM e assim foram feitos dois testes subjetivos exibidos na Tabela \ref{tbl:drr-exp}:
o primeiro, representado pela coluna “Comparação”, identifica qual das AVRs (a convoluída com RIRO ou a convoluída RIRSM) está mais distante;
já o segundo, representado pela coluna “Ordem”, identifica a ordem de mais para menos distante entre as AVRs geradas com RIRSMs.

\begin{table} [H]
    \centering
    \caption{Análise subjetiva de distância.}
    \label{tbl:drr-exp}
    \begin{tabular}{c|c|c|c|c}

        \textbf{Exemplo} & 
        \textbf{$DRR_{org}$ (dB)} & 
        \textbf{$DRR_{res}$ (dB)} & 
        \textbf{Comparação} &
        \textbf{Ordem} \\
        \hline 

        D2 & 4,7 & -2 & simulado & 1º \\
        D1 & -4,5 & 10 & original & 2º \\
        D3 & 0,5 & 18 & original & 3º \\

    \end{tabular}
\end{table}

De acordo com a Tabela \ref{tbl:drr-exp}, foi possível identificar precisamente a diferença de variações do DRR para as RIRSMs.
Na coluna "Comparação" foi identificado que as RIRs com o menor DRR foram avaliadas subjetivamente como mais distantes.
Na coluna "Ordem" as RIRSMs foram ordenadas conforme o esperado, ou seja, do menor para o maior $DRR_{res}$, pois um DRR menor reflete uma sensação
de maior distância.


\subsection{\textit{Data Augmentation} do T60}

Esta seção apresenta os resultados da \textit{Data Augmentation} do T60. Para isso, foram gerados três exemplos de RIRSMs. Suas configurações
são exibidas na Tabela \ref{tbl:da-t60}, onde $T60_{org}$ representa o T60 da RIR original, $T60_{alvo}$ o valor de T60 desejado pelo usuário,
$T60_{res}$ o valor de T60 resultante após DA e $\rho_{T60}$ é o erro definido da forma $\rho_{T60} = abs(T60_{res} - T60_{alvo})/T60_{alvo}$.

\begin{table} [H]
    \centering
    \caption{Exemplos de DA de T60 gerados.}
    \label{tbl:da-t60}
    \begin{tabular}{c|c|c|c}

        \textbf{Exemplo} & 
        \textbf{Sala RIR} & 
        \textbf{Distância (m)} &
        \textbf{Amostra de Voz} \\
        \hline 

        T1 & lecture & 7.1 & M2-T1 \\
        T2 & booth & 1 & H1-T2 \\
        T3 & office & 2 & H2-T2 \\

    \end{tabular}
    \bigbreak
    \bigbreak
    \begin{tabular}{c|c|c|c|c}

        \textbf{Exemplo} & 
        \textbf{$T60_{org}$ (s)} & 
        \textbf{$T60_{alvo}$ (s)} &
        \textbf{$T60_{res}$ (s)} & 
        \textbf{$\rho_{T60}$ (\%)} \\
        \hline 

        T1 & 1,38 & 1,15 & 1,01 & 12.1 \\
        T2 & 1,01 & 1,88 & 1,89 & 0,5 \\
        T3 & 0,75 & 0,61 & 0,60 & 1,6 \\

    \end{tabular}
\end{table}

Nos exemplos gerados, na faixa determinada para o $T60_{alvo}$, foi observada uma mínima diferença entre o $T60_{alvo}$ e o $T60_{res}$ no exemplo T2 e T3.
Para o exemplo T1, notamos um erro considerável ao tentar reduzir o T60, o algoritmo proposto tem melhor acurácia para pequenas variações de T60 no caso de
redução, contudo o mesmo não é observado para o aumento do T60, mesmo com grandes variações entre $T60_{alvo}$ e $T60_{res}$.
É possível observar na Figura \ref{fig:rir-aug-t2} que ocorreu um alongamento da queda exponencial, ou seja, a queda de energia da RIR
acontece de forma mais lenta, na seção equivalente a $h_l(t)$ comparada à Figura \ref{fig:rir-og-t2} para o exemplo T2.

\begin{figure} [H]
    \centering
    \includegraphics[scale=0.25]{rir-og-t2.png}
    \caption{RIR original do exemplo T2.}
    \label{fig:rir-og-t2}
\end{figure} 

\begin{figure} [H]
    \centering
    \includegraphics[scale=0.25]{rir-aug-t2.png}
    \caption{RIR simulada do exemplo T2.}
    \label{fig:rir-aug-t2}
\end{figure} 

De forma análoga aos resultados de DRR, foi feito um experimento subjetivo com o objetivo de avaliar a percepção de “reverberação” do som, ou seja,
a sensação subjetiva do locutor estar falando em um espaço fechado mais amplo.
Foram geradas amostras de voz reverberadas (AVR) usando a RIRO e RIRSM, e assim foram feitos dois testes subjetivos exibidos na Tabela \ref{tbl:t60-exp}:
o primeiro, representado pela coluna “Comparação”, identifica qual das AVRs (a convoluída com RIRO ou a convoluída RIRSM) está com mais reverberação.
E o segundo, representado pela coluna “Ordem”, identifica a ordem ,entre as AVRs geradas com RIRSMs, de mais para menos reverberante.

\begin{table} [H]
    \centering
    \caption{Análise subjetiva de eco.}
    \label{tbl:t60-exp}
    \begin{tabular}{c|c|c|c|c}

        \textbf{Exemplo} & 
        \textbf{$T60_{org}$ (s)} & 
        \textbf{$T60_{res}$ (s)} & 
        \textbf{Comparação} &
        \textbf{Ordem} \\
        \hline 

        T2 & 1,01 & 1,89 & simulado & 1º \\
        T1 & 1,38 & 1,01 & original & 2º \\
        T3 & 0,75 & 0,60 & original & 3º \\

    \end{tabular}
\end{table}

De acordo com a Tabela \ref{tbl:t60-exp}, também foi possível identificar precisamente a diferença de variações do T60 para as RIRSMs.
Na coluna "Comparação" foi identificado que as RIRs com o maior T60 foram avaliadas subjetivamente como mais reverberantes.
Na coluna "Ordem" as RIRSMs foram ordenadas conforme o esperado, ou seja, do maior para o menor $T60_{res}$, pois um T60 maior reflete a sensação
de maior reverberação.

Este resultado demonstra que a DA está ocorrendo, mesmo obtendo o $\rho_{T60}$ elevado para T1.
A discrepância entre $T60_{alvo}$ e $T60_{res}$ pode ser inferida pelas diferenças de implementação entre a técnica de DA descrita no artigo 
\cite{RIR_Data_Aug} e a técnica apresentada neste projeto. 

%mesmo ocorrendo o $\rho$ elevado para T1, indicando que a DA está ocorrendo, mesmo não atingindo o $T60_{alvo}$.

\subsection{\textit{Data Augmentation} de fala em campo distante}

Por último, esta seção apresenta os resultados da \textit{Data Augmentation} de AVCDs. Para isso, 
foram gerados cinco exemplos de AVCDs. Suas configurações são exibidas na Tabela \ref{tbl:da-noise}, onde além dos parâmetros descritos nas
seções anteriores, é exibido o $SNR_{alvo}$ que representa a razão SNR desejada entre a AVCD gerada e os SRP e SRF inseridos.

\begin{table} [H]
    \centering
    \caption{Exemplos de DA de AVCD gerados.}
    \label{tbl:da-noise}
    \begin{tabular}{c|c|c|c|c|c}

        \textbf{Exemplo} & 
        \textbf{Sala RIR} & 
        \textbf{Distância (m)} &
        \textbf{AVA} &
        \textbf{SRP} &
        \textbf{SRF} \\
        \hline 

        N1 & lecture & 7.1 & M2-T1 & RP-6 & RF-1 \\
        N2 & booth & 1 & H2-T1 & RP-12 & RF-4 \\
        N3 & office & 2 & H1-T1 & RP-4 & RF-4 \\
        N4 & meeting & 1.7 & M1-T2 & RP-11 & RF-2 \\
        N5 & stairway & 1 & H2-T1 & RP-7 & RF-4 \\

    \end{tabular}
    \bigbreak
    \bigbreak
    \begin{tabular}{c|c|c|c|c|c}

        \textbf{Exemplo} & 
        \textbf{$DRR_{org}$ (dB)} & 
        \textbf{$DRR_{res}$ (dB)} & 
        \textbf{$T60_{org}$ (s)} & 
        \textbf{$T60_{res}$ (s)} &
        \textbf{$SNR_{alvo}$} \\
        \hline 

        N1 & -4,5 & 17 & 1,38 & 0,56 & 5 \\
        N2 & 4,7 & 17 & 1,01 & 1,39 & 10 \\
        N3 & 0,5 & 14 & 0,75 & 0,60 & 14 \\
        N4 & 6,0 & 16 & 0,81 & 1,16 & 19 \\
        N5 & 5,0 & 18 & 2,70 & 3,68 & 3 \\

    \end{tabular}
\end{table}

Abaixo temos os gráficos relativos ao exemplo N2, contendo a amostra de voz original na Figura \ref{fig:voice-og-n2}, a amostra de voz reverberada
com a RIRSM na Figura \ref{fig:voice-aug-n2} e a amostra de voz em campo distante na Figura \ref{fig:voice-ns-n2}.

\begin{figure} [H]
    \centering
    \includegraphics[scale=0.25]{voice-og-n2.png}
    \caption{Amostra de voz original no exemplo N2.}
    \label{fig:voice-og-n2}
\end{figure} 

\begin{figure} [H]
    \centering
    \includegraphics[scale=0.25]{voice-aug-n2.png}
    \caption{Amostra de voz reverberada com RIRSM no exemplo N2.}
    \label{fig:voice-aug-n2}
\end{figure} 

\begin{figure} [H]
    \centering
    \includegraphics[scale=0.25]{voice-ns-n2.png}
    \caption{Amostra de voz em campo distante no exemplo N2.}
    \label{fig:voice-ns-n2}
\end{figure} 

Conforme o esperado, observa-se que a Figura \ref{fig:voice-ns-n2} possui um ruído residual claramente visível comparado à Figura \ref{fig:voice-aug-n2}.
Nota-se também que na faixa de tempo entre 2000 e 4000 há picos sonoros que não são observados na amostra reverberada, os quais correspondem ao ruído pontual
introduzido no sinal.

De forma análoga, foi feito outro experimento subjetivo com o objetivo de avaliar a percepção de ruído do som, ou seja, 
a sensação subjetiva de dificuldade de entender a fala do locutor devido aos outros sons misturados na amostra de voz.
Foram usadas as cinco AVCDs apresentadas nesta seção e o teste subjetivo é exibido na Tabela \ref{tbl:noise-exp}, onde
a coluna “Ordem” identifica a ordem dos sons de mais para menos ruidoso entre as AVCDs geradas.

\begin{table} [H]
    \centering
    \caption{Análise subjetiva de nível de ruído.}
    \label{tbl:noise-exp}
    \begin{tabular}{c|c|c}

        \textbf{Exemplo} & 
        \textbf{$SNR_{alvo}$ (s)} & 
        \textbf{Ordem} \\
        \hline 

        N3 & 14 & 1º \\
        N5 &  3 & 2º \\
        N1 &  5 & 3º \\
        N2 & 10 & 4º \\
        N4 & 19 & 5º \\
        

    \end{tabular}
\end{table}

Observa-se que, os ruídos foram ordenados corretamente de nível decrescente de ruído, onde a única exceção foi o exemplo N3 votado como 
o mais ruidoso. De acordo com a pessoa que realizou os testes, o exemplo N3 possui um ruído pontual
de longa duração (neste caso representado pelo ruído “porta abrindo”), e isso atrapalhou no reconhecimento da fala do locutor.


% ---------------------------------------------------------------
% Chapter 7 - Conclusões
% ---------------------------------------------------------------
\chapter{Conclusões}
\label{cap7}
Neste trabalho foram propostos dois algoritmos de aumento de dados com o objetivo de gerar uma base de dados de amostras de voz 
em campo distante e RIRSMs para treinamento de redes de \textit{deep learning}.
Para isso, foi necessário avaliar as principais características e modelos usados nas RIRs para deduzir formas de realizar
a modificação das mesmas. Este projeto foi baseado nas técnicas propostas em \cite{RIR_Data_Aug} para DA de RIRs e em \cite{Speech_Rec}
para DA de AVCDs.

Ao final do trabalho, foram obtidas diversas RIRSMs e AVCDs geradas através dos algoritmos propostos. 
Em grande parte, os resultados alcançados estão condizentes com os valores esperados, ou seja, os valores
escolhidos durante a geração dos dados. Foi observado uma discrepância considerável entre os valores de T60 modificados 
para valores abaixo do T60 da RIR original, podendo esta variação ser explicada devido às diferenças de implementação 
do algoritmo de DA de T60 usados em \cite{RIR_Data_Aug} e o proposto neste projeto.

Quanto às conclusões que podem ser inferidas através dos resultados, nota-se que é possível realizar um eficaz
aumento de dados de RIRs e AVCDs, mesmo considerando as discrepâncias com os resultados de T60 obtidos,
pois foi constatado empiricamente que as variações de “distância” e “eco” são perceptíveis e condizentes com as modificações
esperadas.

Para trabalhos futuros, destaca-se a implementação de uma metodologia de aumento de dados da característica do T60
da RIR que mais se aproxima ao que foi usado em \cite{RIR_Data_Aug}. Neste tópico, seria interessante usar outro modelo
de estimativa do T60 e assim observar se há redução nessa discrepância mencionada.

Outra abordagem de trabalho futuro seria comparar os resultados obtidos com as RIRs geradas com o método de DA implementado e RIRs 
geradas com programas de simulação acústicas, como o RAIOS \cite{RAIOS}.

Também seria interessante propor um modelo de rede de \textit{deep learning} para estimação de T60 e DRR em AVCDs,
realizando dois treinos: um com as RIRSMs e AVCDs geradas pelo algoritmo deste trabalho e outro somente com RIRs e AVCDs reais e 
assim observar a eficácia da base de dados gerada artificialmente para treinamentos de redes.









% ---------------------------------------------------------------
% Bibliografia
% ---------------------------------------------------------------
\normalsize
\cleardoublepage
\addcontentsline{toc}{chapter}{Bibliografia}
\bibliographystyle{coppe} 
\bibliography{biblio}

% ---------------------------------------------------------------
% Apêndices 
% ---------------------------------------------------------------
   \appendix
   % ---------------------------------------------------------------
   % Apêndice A
   % ---------------------------------------------------------------
   \chapter{Gráficos dos Resultados}
   \label{ApendiceA}
   \section{\textit{Data Augmentation de DRR}}

\begin{table} [H]
    \centering
    \caption{Exemplos de DA de DRR gerados.}
    \label{tbl-a:da-drr}
    \begin{tabular}{c|c|c|c}

        \textbf{Exemplo} & 
        \textbf{Sala RIR} & 
        \textbf{Distância (m)} &
        \textbf{Amostra de Voz} \\
        \hline 

        D1 & lecture & 7.1 & H2-T2 \\
        D2 & booth & 1 & H2-T1 \\
        D3 & office & 2 & M2-T2 \\

    \end{tabular}
    \bigbreak
    \bigbreak
    \begin{tabular}{c|c|c|c|c}

        \textbf{Exemplo} & 
        \textbf{$DRR_{org}$ (db)} & 
        \textbf{$DRR_{alvo}$ (db)} &
        \textbf{$DRR_{res}$ (db)} & 
        \textbf{$\rho$ (\%)} \\
        \hline 

        D1 & -4,5 & 10 & 10 & 0 \\
        D2 & 4,7 & -2 & -2 & 0 \\
        D3 & 0,5 & 18 & 18 & 0 \\

    \end{tabular}
\end{table}

\pagebreak
{\Large \textbf{Exemplo D1}}

\begin{figure} [H]
    \centering
    \includegraphics[scale=0.25]{rir-og-d1.png}
    \caption{RIR original.}
    \label{fig-a:rir-og-d1}
\end{figure} 

\begin{figure} [H]
    \centering
    \includegraphics[scale=0.25]{rir-aug-d1.png}
    \caption{RIR simulada.}
    \label{fig-a:rir-aug-d1}
\end{figure} 

\begin{figure} [H]
    \centering
    \includegraphics[scale=0.25]{voice-og-d1.png}
    \caption{Amostra de voz original.}
    \label{fig-a:voice-og-d1}
\end{figure} 

\begin{figure} [H]
    \centering
    \includegraphics[scale=0.25]{voice-aug-d1.png}
    \caption{Amostra de voz reverberada com RIRSM.}
    \label{fig-a:voice-aug-d1}
\end{figure}

\pagebreak
{\Large \textbf{Exemplo D2}}

\begin{figure} [H]
    \centering
    \includegraphics[scale=0.25]{rir-og-d2.png}
    \caption{RIR original.}
    \label{fig-a:rir-og-d2}
\end{figure} 

\begin{figure} [H]
    \centering
    \includegraphics[scale=0.25]{rir-aug-d2.png}
    \caption{RIR simulada.}
    \label{fig-a:rir-aug-d2}
\end{figure} 

\begin{figure} [H]
    \centering
    \includegraphics[scale=0.25]{voice-og-d2.png}
    \caption{Amostra de voz original.}
    \label{fig-a:voice-og-d2}
\end{figure} 

\begin{figure} [H]
    \centering
    \includegraphics[scale=0.25]{voice-aug-d2.png}
    \caption{Amostra de voz reverberada com RIRSM.}
    \label{fig-a:voice-aug-d2}
\end{figure}

\pagebreak
{\Large \textbf{Exemplo D3}}

\begin{figure} [H]
    \centering
    \includegraphics[scale=0.25]{rir-og-d3.png}
    \caption{RIR original.}
    \label{fig-a:rir-og-d3}
\end{figure} 

\begin{figure} [H]
    \centering
    \includegraphics[scale=0.25]{rir-aug-d3.png}
    \caption{RIR simulada.}
    \label{fig-a:rir-aug-d3}
\end{figure} 

\begin{figure} [H]
    \centering
    \includegraphics[scale=0.25]{voice-og-d3.png}
    \caption{Amostra de voz original.}
    \label{fig-a:voice-og-d3}
\end{figure} 

\begin{figure} [H]
    \centering
    \includegraphics[scale=0.25]{voice-aug-d3.png}
    \caption{Amostra de voz reverberada com RIRSM.}
    \label{fig-a:voice-aug-d3}
\end{figure}

\pagebreak
\section{\textit{Data Augmentation de T60}}

\begin{table} [H]
    \centering
    \caption{Exemplos de DA de T60 gerados.}
    \label{tbl-a:da-t60}
    \begin{tabular}{c|c|c|c}

        \textbf{Exemplo} & 
        \textbf{Sala RIR} & 
        \textbf{Distância (m)} &
        \textbf{Amostra de Voz} \\
        \hline 

        T1 & lecture & 7.1 & M2-T1 \\
        T2 & booth & 1 & H1-T2 \\
        T3 & office & 2 & H2-T2 \\

    \end{tabular}
    \bigbreak
    \bigbreak
    \begin{tabular}{c|c|c|c|c}

        \textbf{Exemplo} & 
        \textbf{$T60_{org}$ (s)} & 
        \textbf{$T60_{alvo}$ (s)} &
        \textbf{$T60_{res}$ (s)} & 
        \textbf{$\rho$ (\%)} \\
        \hline 

        T1 & 1,38 & 1,15 & 1,01 & 12.1 \\
        T2 & 1,01 & 1,88 & 1,89 & 0,5 \\
        T3 & 0,75 & 0,61 & 0,60 & 1,6 \\

    \end{tabular}
\end{table}

\pagebreak
{\Large \textbf{Exemplo T1}}

\begin{figure} [H]
    \centering
    \includegraphics[scale=0.25]{rir-og-t1.png}
    \caption{RIR original.}
    \label{fig-a:rir-og-t1}
\end{figure} 

\begin{figure} [H]
    \centering
    \includegraphics[scale=0.25]{rir-aug-t1.png}
    \caption{RIR simulada.}
    \label{fig-a:rir-aug-t1}
\end{figure} 

\begin{figure} [H]
    \centering
    \includegraphics[scale=0.25]{voice-og-t1.png}
    \caption{Amostra de voz original.}
    \label{fig-a:voice-og-t1}
\end{figure} 

\begin{figure} [H]
    \centering
    \includegraphics[scale=0.25]{voice-aug-t1.png}
    \caption{Amostra de voz reverberada com RIRSM.}
    \label{fig-a:voice-aug-t1}
\end{figure}

\pagebreak
{\Large \textbf{Exemplo T2}}

\begin{figure} [H]
    \centering
    \includegraphics[scale=0.25]{rir-og-t2.png}
    \caption{RIR original.}
    \label{fig-a:rir-og-t2}
\end{figure} 

\begin{figure} [H]
    \centering
    \includegraphics[scale=0.25]{rir-aug-t2.png}
    \caption{RIR simulada.}
    \label{fig-a:rir-aug-t2}
\end{figure} 

\begin{figure} [H]
    \centering
    \includegraphics[scale=0.25]{voice-og-t2.png}
    \caption{Amostra de voz original.}
    \label{fig-a:voice-og-t2}
\end{figure} 

\begin{figure} [H]
    \centering
    \includegraphics[scale=0.25]{voice-aug-t2.png}
    \caption{Amostra de voz reverberada com RIRSM.}
    \label{fig-a:voice-aug-t2}
\end{figure}

\pagebreak
{\Large \textbf{Exemplo T3}}

\begin{figure} [H]
    \centering
    \includegraphics[scale=0.25]{rir-og-t3.png}
    \caption{RIR original.}
    \label{fig-a:rir-og-t3}
\end{figure} 

\begin{figure} [H]
    \centering
    \includegraphics[scale=0.25]{rir-aug-t3.png}
    \caption{RIR simulada.}
    \label{fig-a:rir-aug-t3}
\end{figure} 

\begin{figure} [H]
    \centering
    \includegraphics[scale=0.25]{voice-og-t3.png}
    \caption{Amostra de voz original.}
    \label{fig-a:voice-og-t3}
\end{figure} 

\begin{figure} [H]
    \centering
    \includegraphics[scale=0.25]{voice-aug-t3.png}
    \caption{Amostra de voz reverberada com RIRSM.}
    \label{fig-a:voice-aug-t3}
\end{figure}

\pagebreak
\section{\textit{Data Augmentation de AVCD}}

\begin{table} [H]
    \centering
    \caption{Exemplos de DA de AVCD gerados.}
    \label{tbl-a:da-noise}
    \begin{tabular}{c|c|c|c|c|c}

        \textbf{Exemplo} & 
        \textbf{Sala RIR} & 
        \textbf{Distância (m)} &
        \textbf{AVA} &
        \textbf{SRP} &
        \textbf{SRF} \\
        \hline 

        N1 & lecture & 7.1 & M2-T1 & RP-6 & RF-1 \\
        N2 & booth & 1 & H2-T1 & RP-12 & RF-4 \\
        N3 & office & 2 & H1-T1 & RP-4 & RF-4 \\
        N4 & meeting & 1.7 & M1-T2 & RP-11 & RF-2 \\
        N5 & stairway & 1 & H2-T1 & RP-7 & RF-4 \\

    \end{tabular}
    \bigbreak
    \bigbreak
    \begin{tabular}{c|c|c|c|c|c}

        \textbf{Exemplo} & 
        \textbf{$DRR_{org}$ (db)} & 
        \textbf{$DRR_{res}$ (db)} & 
        \textbf{$T60_{org}$ (s)} & 
        \textbf{$T60_{res}$ (s)} &
        \textbf{$SNR_{alvo}$} \\
        \hline 

        N1 & -4,5 & 17 & 1,38 & 0,56 & 5 \\
        N2 & 4,7 & 17 & 1,01 & 1,39 & 10 \\
        N3 & 0,5 & 14 & 0,75 & 0,60 & 14 \\
        N4 & 6,0 & 16 & 0,81 & 1,16 & 19 \\
        N5 & 5,0 & 18 & 2,70 & 3,68 & 3 \\

    \end{tabular}
\end{table}

\pagebreak
{\Large \textbf{Exemplo N1}}

\begin{figure} [H]
    \centering
    \includegraphics[scale=0.25]{rir-og-n1.png}
    \caption{RIR original.}
    \label{fig-a:rir-og-n1}
\end{figure} 

\begin{figure} [H]
    \centering
    \includegraphics[scale=0.25]{rir-aug-n1.png}
    \caption{RIR simulada.}
    \label{fig-a:rir-aug-n1}
\end{figure} 

\begin{figure} [H]
    \centering
    \includegraphics[scale=0.25]{voice-og-n1.png}
    \caption{Amostra de voz original.}
    \label{fig-a:voice-og-n1}
\end{figure} 

\begin{figure} [H]
    \centering
    \includegraphics[scale=0.25]{voice-aug-n1.png}
    \caption{Amostra de voz reverberada com RIRSM.}
    \label{fig-a:voice-aug-n1}
\end{figure}

\begin{figure} [H]
    \centering
    \includegraphics[scale=0.25]{voice-ns-n1.png}
    \caption{Amostra de voz em campo distante.}
    \label{fig-a:voice-ns-n1}
\end{figure}

\pagebreak
{\Large \textbf{Exemplo N2}}

\begin{figure} [H]
    \centering
    \includegraphics[scale=0.25]{rir-og-n2.png}
    \caption{RIR original.}
    \label{fig-a:rir-og-n2}
\end{figure} 

\begin{figure} [H]
    \centering
    \includegraphics[scale=0.25]{rir-aug-n2.png}
    \caption{RIR simulada.}
    \label{fig-a:rir-aug-n2}
\end{figure} 

\begin{figure} [H]
    \centering
    \includegraphics[scale=0.25]{voice-og-n2.png}
    \caption{Amostra de voz original.}
    \label{fig-a:voice-og-n2}
\end{figure} 

\begin{figure} [H]
    \centering
    \includegraphics[scale=0.25]{voice-aug-n2.png}
    \caption{Amostra de voz reverberada com RIRSM.}
    \label{fig-a:voice-aug-n2}
\end{figure}

\begin{figure} [H]
    \centering
    \includegraphics[scale=0.25]{voice-ns-n2.png}
    \caption{Amostra de voz em campo distante.}
    \label{fig-a:voice-ns-n2}
\end{figure}

\pagebreak
{\Large \textbf{Exemplo N3}}

\begin{figure} [H]
    \centering
    \includegraphics[scale=0.25]{rir-og-n3.png}
    \caption{RIR original.}
    \label{fig-a:rir-og-n3}
\end{figure} 

\begin{figure} [H]
    \centering
    \includegraphics[scale=0.25]{rir-aug-n3.png}
    \caption{RIR simulada.}
    \label{fig-a:rir-aug-n3}
\end{figure} 

\begin{figure} [H]
    \centering
    \includegraphics[scale=0.25]{voice-og-n3.png}
    \caption{Amostra de voz original.}
    \label{fig-a:voice-og-n3}
\end{figure} 

\begin{figure} [H]
    \centering
    \includegraphics[scale=0.25]{voice-aug-n3.png}
    \caption{Amostra de voz reverberada com RIRSM.}
    \label{fig-a:voice-aug-n3}
\end{figure}

\begin{figure} [H]
    \centering
    \includegraphics[scale=0.25]{voice-ns-n3.png}
    \caption{Amostra de voz em campo distante.}
    \label{fig-a:voice-ns-n3}
\end{figure}

\pagebreak
{\Large \textbf{Exemplo N4}}

\begin{figure} [H]
    \centering
    \includegraphics[scale=0.25]{rir-og-n4.png}
    \caption{RIR original.}
    \label{fig-a:rir-og-n4}
\end{figure} 

\begin{figure} [H]
    \centering
    \includegraphics[scale=0.25]{rir-aug-n4.png}
    \caption{RIR simulada.}
    \label{fig-a:rir-aug-n4}
\end{figure} 

\begin{figure} [H]
    \centering
    \includegraphics[scale=0.25]{voice-og-n4.png}
    \caption{Amostra de voz original.}
    \label{fig-a:voice-og-n4}
\end{figure} 

\begin{figure} [H]
    \centering
    \includegraphics[scale=0.25]{voice-aug-n4.png}
    \caption{Amostra de voz reverberada com RIRSM.}
    \label{fig-a:voice-aug-n4}
\end{figure}

\begin{figure} [H]
    \centering
    \includegraphics[scale=0.25]{voice-ns-n4.png}
    \caption{Amostra de voz em campo distante.}
    \label{fig-a:voice-ns-n4}
\end{figure}

\pagebreak
{\Large \textbf{Exemplo N5}}

\begin{figure} [H]
    \centering
    \includegraphics[scale=0.25]{rir-og-n5.png}
    \caption{RIR original.}
    \label{fig-a:rir-og-n5}
\end{figure} 

\begin{figure} [H]
    \centering
    \includegraphics[scale=0.25]{rir-aug-n5.png}
    \caption{RIR simulada.}
    \label{fig-a:rir-aug-n5}
\end{figure} 

\begin{figure} [H]
    \centering
    \includegraphics[scale=0.25]{voice-og-n5.png}
    \caption{Amostra de voz original.}
    \label{fig-a:voice-og-n5}
\end{figure} 

\begin{figure} [H]
    \centering
    \includegraphics[scale=0.25]{voice-aug-n5.png}
    \caption{Amostra de voz reverberada com RIRSM.}
    \label{fig-a:voice-aug-n5}
\end{figure}

\begin{figure} [H]
    \centering
    \includegraphics[scale=0.25]{voice-ns-n5.png}
    \caption{Amostra de voz em campo distante.}
    \label{fig-a:voice-ns-n5}
\end{figure}
   % ---------------------------------------------------------------
   % Apêndice B
   % ---------------------------------------------------------------
   \chapter{Código Fonte}
   \label{ApendiceB}
   O código fonte da implementação da metodologia deste trabalho está disponível no \textit{GitHub}.
Segue o link abaixo.

https://github.com/afonsobm/RIR-Augmentation-MATLAB
   % % ---------------------------------------------------------------
   % % Apêndice C
   % % ---------------------------------------------------------------
   % \chapter{O que é um anexo?}
   % \label{ApendiceC}
   % \input{ApendiceC}   

\backmatter

\end{document}
