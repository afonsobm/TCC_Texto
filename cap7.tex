Neste trabalho foram propostos dois algoritmos de \textit{data augmentation} com o objetivo de gerar uma base de dados de amostras de voz 
em campo distante e RIRSMs para treinamento de redes de \textit{deep learning}.
Para isso, foi necessário avaliar as principais características e modelos usados nas RIRs para deduzir formas de realizar
a modificação das mesmas. Este projeto foi baseado nas técnicas propostas em \cite{RIR_Data_Aug} para DA de RIRs e em \cite{Speech_Rec}
para DA de AVCDs.

Ao final do trabalho, foram obtidas diversas RIRSMs e AVCDs geradas através dos algoritmos propostos. 
Em grande parte, os resultados alcançados estão condizentes com os valores esperados, ou seja, os valores
escolhidos durante a geração dos dados. Foi observado uma discrepância considerável entre os valores de T60 modificados 
para valores abaixo do T60 da RIR original, podendo esta variação ser explicada devido às diferenças de implementação 
do algoritmo de DA de T60 usados em \cite{RIR_Data_Aug} e o proposto neste projeto.

Quanto às conclusões que podem ser inferidas através dos resultados, nota-se que é possível realizar uma eficaz
\textit{data augmentation} de RIRs e AVCDs, mesmo considerando as discrepâncias com os resultados de T60 obtidos,
pois foi constatado empiricamente que as variações de “distância” e “reverberação” são perceptíveis e condizentes com as modificações
esperadas.

Para trabalhos futuros, destaca-se a implementação de uma metodologia de \textit{data augmentation} da característica do T60
da RIR que mais se aproxima ao que foi usado em \cite{RIR_Data_Aug}. Neste tópico, seria interessante usar outro modelo
de estimativa do T60 e assim observar se há redução nessa discrepância mencionada.

Outra abordagem de trabalho futuro seria comparar os resultados obtidos com as RIRs geradas com o método de DA implementado e RIRs 
geradas com programas de simulação acústicas, como o RAIOS \cite{RAIOS}.

Também seria interessante propor um modelo de rede de \textit{deep learning} para estimação de T60 e DRR em AVCDs,
realizando dois treinos: um com as RIRSMs e AVCDs geradas pelo algoritmo deste trabalho e outro somente com RIRs e AVCDs reais e 
assim observar a eficácia da base de dados gerada artificialmente para treinamentos de redes.







