Este capítulo descreve os procedimentos e as configurações feitas para a implementação empírica da metodologia deste trabalho.
Cada seção exibe um conjunto de gráficos relativos ao procedimento implementado para um único exemplo. Os gráficos obtidos para todos
os exemplos se encontram no Apêndice \ref{ApendiceA}.
O código fonte da implementação dos algoritmos propostos se encontra no Apêndice \ref{ApendiceB}.

\section{Configuração dos parâmetros}

As RIRSMs e AVCDs geradas nestes resultados usam as bases de dados de RIR, amostras de voz anecóicas e ruídos descritas no Capítulo \ref{cap5}.
As faixas de DRR, T60 e SNR usados são exibidos na Tabela \ref{tbl:config-param}.

\begin{table} [H]
    \centering
    \caption{Faixas dos parâmetros.}
    \label{tbl:config-param}
    \begin{tabular}{c|p{9cm}}

        \multicolumn{1}{c|}{\textbf{Parâmetro}} & \multicolumn{1}{c}{\textbf{Faixa}} \\
        \hline 

        $DRR_{alvo}$ (dB) & $-6 \le DRR_{alvo} \le 18 $ \\
        $T60_{alvo}$ (s) & $T60_{org} - 1  \le T60_{alvo} \le T60_{org} + 1$ , onde o limite inferior de $T60_{alvo} = 0.2$ \\
        $SNR_{alvo}$ & $3 \le SNR_{alvo} \le 20 $ \\

    \end{tabular}
\end{table}

O $DRR_{alvo}$ segue os valores propostos no artigo \cite{RIR_Data_Aug}, já os valores de $T60_{alvo}$, devido à maior faixa de valores 
de $T60$ das RIRs base de AIR, são limitados a $\pm 1 $ segundo comparado ao valor de $T60_{org}$ da RIR original usada como base para o algoritmo
de DA implementado.

\section{Resultados}

\subsection{\textit{Data Augmentation} do DRR}

Esta seção apresenta os resultados da \textit{Data Augmentation} do DRR. Para isso, foram gerados três exemplos de RIRSMs. Suas configurações
são exibidas na Tabela \ref{tbl:da-drr}, onde $DRR_{org}$ representa o DRR da RIR original, $DRR_{alvo}$ o valor de DRR desejado pelo usuário,
$DRR_{res}$ o valor de DRR resultante após DA e $\rho_{DRR}$ é o erro relativo definido da forma $\rho_{DRR} = abs(DRR_{res} - DRR_{alvo})/DRR_{alvo}$.

\begin{table} [H]
    \centering
    \caption{Exemplos de DA de DRR gerados.}
    \label{tbl:da-drr}
    \begin{tabular}{c|c|c|c}

        \textbf{Exemplo} & 
        \textbf{Sala RIR} & 
        \textbf{Distância (m)} &
        \textbf{Amostra de Voz} \\
        \hline 

        D1 & lecture & 7.1 & H2-T2 \\
        D2 & booth & 1 & H2-T1 \\
        D3 & office & 2 & M2-T2 \\

    \end{tabular}
    \bigbreak
    \bigbreak
    \begin{tabular}{c|c|c|c|c}

        \textbf{Exemplo} & 
        \textbf{$DRR_{org}$ (dB)} & 
        \textbf{$DRR_{alvo}$ (dB)} &
        \textbf{$DRR_{res}$ (dB)} & 
        \textbf{$\rho_{DRR}$ (\%)} \\
        \hline 

        D1 & -4,5 & 10 & 10 & 0 \\
        D2 & 4,7 & -2 & -2 & 0 \\
        D3 & 0,5 & 18 & 18 & 0 \\

    \end{tabular}
\end{table}


Nos exemplos gerados, na faixa determinada para o $DRR_{alvo}$, não houve diferença entre o $DRR_{alvo}$ e o $DRR_{res}$.
É possível observar na Figura \ref{fig:rir-aug-d1} que ocorreu um aumento na seção equivalente a $h_e(t)$ comparado à da Figura \ref{fig:rir-og-d1}
para o exemplo D1.

\begin{figure} [H]
    \centering
    \includegraphics[scale=0.25]{rir-og-d1.png}
    \caption{RIR original do exemplo D1.}
    \label{fig:rir-og-d1}
\end{figure} 

\begin{figure} [H]
    \centering
    \includegraphics[scale=0.25]{rir-aug-d1.png}
    \caption{RIR simulada do exemplo D1.}
    \label{fig:rir-aug-d1}
\end{figure} 

Foi realizado também um experimento subjetivo com uma pessoa sem problemas auditivos, cuja idade é de 25 anos na data do experimento,  
que não possui prévio conhecimento dos resultados gerados.
A pessoa foi colocada em um quarto silencioso e foi usado um fone de ouvido para ouvir os aúdios gerados para este experimento.
O objetivo é avaliar a percepção de “distância” do som, ou seja, a sensação subjetiva do locutor estar falando próximo ou distante
do microfone.
Foram geradas amostras de voz reverberadas (AVR) usando a RIRO e RIRSM e assim foram feitos dois testes subjetivos exibidos na Tabela \ref{tbl:drr-exp}:
o primeiro, representado pela coluna “Comparação”, identifica qual das AVRs (a convoluída com RIRO ou a convoluída RIRSM) está mais distante;
já o segundo, representado pela coluna “Ordem”, identifica a ordem de mais para menos distante entre as AVRs geradas com RIRSMs.

\begin{table} [H]
    \centering
    \caption{Análise subjetiva de distância.}
    \label{tbl:drr-exp}
    \begin{tabular}{c|c|c|c|c}

        \textbf{Exemplo} & 
        \textbf{$DRR_{org}$ (dB)} & 
        \textbf{$DRR_{res}$ (dB)} & 
        \textbf{Comparação} &
        \textbf{Ordem} \\
        \hline 

        D2 & 4,7 & -2 & simulado & 1º \\
        D1 & -4,5 & 10 & original & 2º \\
        D3 & 0,5 & 18 & original & 3º \\

    \end{tabular}
\end{table}

De acordo com a Tabela \ref{tbl:drr-exp}, foi possível identificar precisamente a diferença de variações do DRR para as RIRSMs.
Na coluna "Comparação" foi identificado que as RIRs com o menor DRR foram avaliadas subjetivamente como mais distantes.
Na coluna "Ordem" as RIRSMs foram ordenadas conforme o esperado, ou seja, do menor para o maior $DRR_{res}$, pois um DRR menor reflete uma sensação
de maior distância.


\subsection{\textit{Data Augmentation} do T60}

Esta seção apresenta os resultados da \textit{Data Augmentation} do T60. Para isso, foram gerados três exemplos de RIRSMs. Suas configurações
são exibidas na Tabela \ref{tbl:da-t60}, onde $T60_{org}$ representa o T60 da RIR original, $T60_{alvo}$ o valor de T60 desejado pelo usuário,
$T60_{res}$ o valor de T60 resultante após DA e $\rho_{T60}$ é o erro definido da forma $\rho_{T60} = abs(T60_{res} - T60_{alvo})/T60_{alvo}$.

\begin{table} [H]
    \centering
    \caption{Exemplos de DA de T60 gerados.}
    \label{tbl:da-t60}
    \begin{tabular}{c|c|c|c}

        \textbf{Exemplo} & 
        \textbf{Sala RIR} & 
        \textbf{Distância (m)} &
        \textbf{Amostra de Voz} \\
        \hline 

        T1 & lecture & 7.1 & M2-T1 \\
        T2 & booth & 1 & H1-T2 \\
        T3 & office & 2 & H2-T2 \\

    \end{tabular}
    \bigbreak
    \bigbreak
    \begin{tabular}{c|c|c|c|c}

        \textbf{Exemplo} & 
        \textbf{$T60_{org}$ (s)} & 
        \textbf{$T60_{alvo}$ (s)} &
        \textbf{$T60_{res}$ (s)} & 
        \textbf{$\rho_{T60}$ (\%)} \\
        \hline 

        T1 & 1,38 & 1,15 & 1,01 & 12.1 \\
        T2 & 1,01 & 1,88 & 1,89 & 0,5 \\
        T3 & 0,75 & 0,61 & 0,60 & 1,6 \\

    \end{tabular}
\end{table}

Nos exemplos gerados, na faixa determinada para o $T60_{alvo}$, foi observada uma mínima diferença entre o $T60_{alvo}$ e o $T60_{res}$ no exemplo T2 e T3.
Para o exemplo T1, notamos um erro considerável ao tentar reduzir o T60, o algoritmo proposto tem melhor acurácia para pequenas variações de T60 no caso de
redução, contudo o mesmo não é observado para o aumento do T60, mesmo com grandes variações entre $T60_{alvo}$ e $T60_{res}$.
É possível observar na Figura \ref{fig:rir-aug-t2} que ocorreu um alongamento da queda exponencial, ou seja, a queda de energia da RIR
acontece de forma mais lenta, na seção equivalente a $h_l(t)$ comparada à Figura \ref{fig:rir-og-t2} para o exemplo T2.

\begin{figure} [H]
    \centering
    \includegraphics[scale=0.25]{rir-og-t2.png}
    \caption{RIR original do exemplo T2.}
    \label{fig:rir-og-t2}
\end{figure} 

\begin{figure} [H]
    \centering
    \includegraphics[scale=0.25]{rir-aug-t2.png}
    \caption{RIR simulada do exemplo T2.}
    \label{fig:rir-aug-t2}
\end{figure} 

De forma análoga aos resultados de DRR, foi feito um experimento subjetivo com o objetivo de avaliar a percepção de “reverberação” do som, ou seja,
a sensação subjetiva do locutor estar falando em um espaço fechado mais amplo.
Foram geradas amostras de voz reverberadas (AVR) usando a RIRO e RIRSM, e assim foram feitos dois testes subjetivos exibidos na Tabela \ref{tbl:t60-exp}:
o primeiro, representado pela coluna “Comparação”, identifica qual das AVRs (a convoluída com RIRO ou a convoluída RIRSM) está com mais reverberação.
E o segundo, representado pela coluna “Ordem”, identifica a ordem ,entre as AVRs geradas com RIRSMs, de mais para menos reverberante.

\begin{table} [H]
    \centering
    \caption{Análise subjetiva de eco.}
    \label{tbl:t60-exp}
    \begin{tabular}{c|c|c|c|c}

        \textbf{Exemplo} & 
        \textbf{$T60_{org}$ (s)} & 
        \textbf{$T60_{res}$ (s)} & 
        \textbf{Comparação} &
        \textbf{Ordem} \\
        \hline 

        T2 & 1,01 & 1,89 & simulado & 1º \\
        T1 & 1,38 & 1,01 & original & 2º \\
        T3 & 0,75 & 0,60 & original & 3º \\

    \end{tabular}
\end{table}

De acordo com a Tabela \ref{tbl:t60-exp}, também foi possível identificar precisamente a diferença de variações do T60 para as RIRSMs.
Na coluna "Comparação" foi identificado que as RIRs com o maior T60 foram avaliadas subjetivamente como mais reverberantes.
Na coluna "Ordem" as RIRSMs foram ordenadas conforme o esperado, ou seja, do maior para o menor $T60_{res}$, pois um T60 maior reflete a sensação
de maior reverberação.

Este resultado demonstra que a DA está ocorrendo, mesmo obtendo o $\rho_{T60}$ elevado para T1.
A discrepância entre $T60_{alvo}$ e $T60_{res}$ pode ser inferida pelas diferenças de implementação entre a técnica de DA descrita no artigo 
\cite{RIR_Data_Aug} e a técnica apresentada neste projeto. 

%mesmo ocorrendo o $\rho$ elevado para T1, indicando que a DA está ocorrendo, mesmo não atingindo o $T60_{alvo}$.

\subsection{\textit{Data Augmentation} de fala em campo distante}

Por último, esta seção apresenta os resultados da \textit{Data Augmentation} de AVCDs. Para isso, 
foram gerados cinco exemplos de AVCDs. Suas configurações são exibidas na Tabela \ref{tbl:da-noise}, onde além dos parâmetros descritos nas
seções anteriores, é exibido o $SNR_{alvo}$ que representa a razão SNR desejada entre a AVCD gerada e os SRP e SRF inseridos.

\begin{table} [H]
    \centering
    \caption{Exemplos de DA de AVCD gerados.}
    \label{tbl:da-noise}
    \begin{tabular}{c|c|c|c|c|c}

        \textbf{Exemplo} & 
        \textbf{Sala RIR} & 
        \textbf{Distância (m)} &
        \textbf{AVA} &
        \textbf{SRP} &
        \textbf{SRF} \\
        \hline 

        N1 & lecture & 7.1 & M2-T1 & RP-6 & RF-1 \\
        N2 & booth & 1 & H2-T1 & RP-12 & RF-4 \\
        N3 & office & 2 & H1-T1 & RP-4 & RF-4 \\
        N4 & meeting & 1.7 & M1-T2 & RP-11 & RF-2 \\
        N5 & stairway & 1 & H2-T1 & RP-7 & RF-4 \\

    \end{tabular}
    \bigbreak
    \bigbreak
    \begin{tabular}{c|c|c|c|c|c}

        \textbf{Exemplo} & 
        \textbf{$DRR_{org}$ (dB)} & 
        \textbf{$DRR_{res}$ (dB)} & 
        \textbf{$T60_{org}$ (s)} & 
        \textbf{$T60_{res}$ (s)} &
        \textbf{$SNR_{alvo}$} \\
        \hline 

        N1 & -4,5 & 17 & 1,38 & 0,56 & 5 \\
        N2 & 4,7 & 17 & 1,01 & 1,39 & 10 \\
        N3 & 0,5 & 14 & 0,75 & 0,60 & 14 \\
        N4 & 6,0 & 16 & 0,81 & 1,16 & 19 \\
        N5 & 5,0 & 18 & 2,70 & 3,68 & 3 \\

    \end{tabular}
\end{table}

Abaixo temos os gráficos relativos ao exemplo N2, contendo a amostra de voz original na Figura \ref{fig:voice-og-n2}, a amostra de voz reverberada
com a RIRSM na Figura \ref{fig:voice-aug-n2} e a amostra de voz em campo distante na Figura \ref{fig:voice-ns-n2}.

\begin{figure} [H]
    \centering
    \includegraphics[scale=0.25]{voice-og-n2.png}
    \caption{Amostra de voz original no exemplo N2.}
    \label{fig:voice-og-n2}
\end{figure} 

\begin{figure} [H]
    \centering
    \includegraphics[scale=0.25]{voice-aug-n2.png}
    \caption{Amostra de voz reverberada com RIRSM no exemplo N2.}
    \label{fig:voice-aug-n2}
\end{figure} 

\begin{figure} [H]
    \centering
    \includegraphics[scale=0.25]{voice-ns-n2.png}
    \caption{Amostra de voz em campo distante no exemplo N2.}
    \label{fig:voice-ns-n2}
\end{figure} 

Conforme o esperado, observa-se que a Figura \ref{fig:voice-ns-n2} possui um ruído residual claramente visível comparado à Figura \ref{fig:voice-aug-n2}.
Nota-se também que na faixa de tempo entre 2000 e 4000 há picos sonoros que não são observados na amostra reverberada, os quais correspondem ao ruído pontual
introduzido no sinal.

De forma análoga, foi feito outro experimento subjetivo com o objetivo de avaliar a percepção de ruído do som, ou seja, 
a sensação subjetiva de dificuldade de entender a fala do locutor devido aos outros sons misturados na amostra de voz.
Foram usadas as cinco AVCDs apresentadas nesta seção e o teste subjetivo é exibido na Tabela \ref{tbl:noise-exp}, onde
a coluna “Ordem” identifica a ordem dos sons de mais para menos ruidoso entre as AVCDs geradas.

\begin{table} [H]
    \centering
    \caption{Análise subjetiva de nível de ruído.}
    \label{tbl:noise-exp}
    \begin{tabular}{c|c|c}

        \textbf{Exemplo} & 
        \textbf{$SNR_{alvo}$ (s)} & 
        \textbf{Ordem} \\
        \hline 

        N3 & 14 & 1º \\
        N5 &  3 & 2º \\
        N1 &  5 & 3º \\
        N2 & 10 & 4º \\
        N4 & 19 & 5º \\
        

    \end{tabular}
\end{table}

Observa-se que, os ruídos foram ordenados corretamente de nível decrescente de ruído, onde a única exceção foi o exemplo N3 votado como 
o mais ruidoso. De acordo com a pessoa que realizou os testes, o exemplo N3 possui um ruído pontual
de longa duração (neste caso representado pelo ruído “porta abrindo”), e isso atrapalhou no reconhecimento da fala do locutor.
