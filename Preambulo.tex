% Declaracao
\begin{center}
Declaração de Autoria e de Direitos
\end{center}

\vspace{0.5cm}

Eu, \emph{Bruno Machado Afonso} CPF \emph{136.151.347-02}, autor da monografia \emph{título da monografia}, subscrevo para os devidos fins, as seguintes informações:\\
1. O autor declara que o trabalho apresentado na disciplina de Projeto de Graduação da Escola Politécnica da UFRJ é de sua autoria, sendo original em forma e conteúdo.\\
2. Excetuam-se do item 1. eventuais transcrições de texto, figuras, tabelas, conceitos e ideias, que identifiquem claramente a fonte original, explicitando as autorizações obtidas dos respectivos proprietários, quando necessárias.\\
3. O autor permite que a UFRJ, por um prazo indeterminado, efetue em qualquer mídia de divulgação, a publicação do trabalho acadêmico em sua totalidade, ou em parte. Essa autorização não envolve ônus de qualquer natureza à UFRJ, ou aos seus representantes.\\
4. O autor pode, excepcionalmente, encaminhar à Comissão de Projeto de Graduação, a não divulgação do material, por um prazo máximo de 01 (um) ano, improrrogável, a contar da data de defesa, desde que o pedido seja justificado, e solicitado antecipadamente, por escrito, à Congregação da Escola Politécnica.\\
5. O autor declara, ainda, ter a capacidade jurídica para a prática do presente ato, assim como ter conhecimento do teor da presente Declaração, estando ciente das sanções e punições legais, no que tange a cópia parcial, ou total, de obra intelectual, o que se configura como violação do direito autoral previsto no Código Penal Brasileiro no art.184 e art.299, bem como na Lei 9.610.\\
6. O autor é o único responsável pelo conteúdo apresentado nos trabalhos acadêmicos publicados, não cabendo à UFRJ, aos seus representantes,  ou ao(s) orientador(es), qualquer responsabilização/ indenização nesse sentido.\\
7. Por ser verdade, firmo a presente declaração.\\

      \vspace{0.5cm}
      \begin{flushright}
         \parbox{10cm}{
            \hrulefill

            \vspace{-.375cm}
            \centering{Bruno Machado Afonso}

            \vspace{0.1cm}
         }
      \end{flushright}
      
\pagebreak

% Copyright
      \vspace{0.5cm}

UNIVERSIDADE FEDERAL DO RIO DE JANEIRO \\
Escola Politécnica - Departamento de Eletrônica e de Computação \\
Centro de Tecnologia, bloco H, sala H-217, Cidade Universitária \\ 
Rio de Janeiro - RJ      CEP 21949-900\\
\vspace{0.5cm}
\paragraph{}Este exemplar é de propriedade da Universidade Federal do Rio de Janeiro, que poderá incluí-lo em base de dados, armazenar em computador, microfilmar ou adotar qualquer forma de arquivamento.
\paragraph{}É permitida a menção, reprodução parcial ou integral e a transmissão entre bibliotecas deste trabalho, sem modificação de seu texto, em qualquer meio que esteja ou venha a ser fixado, para pesquisa acadêmica, comentários e citações, desde que sem finalidade comercial e que seja feita a referência bibliográfica completa.
\paragraph{}Os conceitos expressos neste trabalho são de responsabilidade do(s) autor(es).

\pagebreak


% Agradecimento
\begin{center}
\textbf{AGRADECIMENTO}
\end{center}
      \vspace{0.5cm}

\paragraph{}Sempre haverá. Se não estiver inspirado, aqui está uma sugestão: dedico este trabalho ao povo brasileiro que contribuiu de forma significativa à minha formação e estada nesta Universidade. Este projeto é uma pequena forma de retribuir o investimento e confiança em mim depositados.

\pagebreak


% Resumo
\begin{center}
\textbf{RESUMO}
\end{center}
      \vspace{0.5cm}

\paragraph{}Inserir o resumo do seu trabalho aqui. O objetivo é apresentar ao pretenso leitor do seu Projeto Final uma descrição genérica do seu trabalho. Você também deve tentar despertar no leitor o interesse pelo conteúdo deste documento.
\paragraph{}
\noindent Palavras-Chave: trabalho, resumo, interesse, projeto final.

\pagebreak


% Abstract
\begin{center}
\textbf{ABSTRACT}
\end{center}
      \vspace{0.5cm}

\paragraph{}Insert your abstract here. Insert your abstract here. Insert your abstract here. Insert your abstract here. Insert your abstract here.
\paragraph{}
\noindent Key-words: word, word, word.

\pagebreak


% Siglas
\begin{center}
\textbf{SIGLAS}
\end{center}
      \vspace{0.5cm}

\paragraph{}DRR - Razão Direto-Reverberante
\paragraph{}RIR - Resposta ao Impulso de Ambiente Acústico
\paragraph{}T20 - Tempo de Reverberação (queda de 20 DB)
\paragraph{}T30 - Tempo de Reverberação (queda de 30 DB)
\paragraph{}T60 - Tempo de Reverberação (queda de 60 DB)
\paragraph{}UFRJ - Universidade Federal do Rio de Janeiro 


\pagebreak







