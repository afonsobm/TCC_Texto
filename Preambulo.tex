% Declaracao
\begin{center}
Declaração de Autoria e de Direitos
\end{center}

\vspace{0.4cm}

Eu, \emph{Bruno Machado Afonso} CPF \emph{136.151.347-02}, autor da monografia \emph{Desenvolvimento de Base de Dados para Treinamento de Redes Neurais de Reconhecimento de Voz Através da Geração de Áudios com Respostas
ao Impulso Simuladas por Técnicas de Data Augmentation}, subscrevo para os devidos fins, as seguintes informações:\\
1. O autor declara que o trabalho apresentado na disciplina de Projeto de Graduação da Escola Politécnica da UFRJ é de sua autoria, sendo original em forma e conteúdo.\\
2. Excetuam-se do item 1. eventuais transcrições de texto, figuras, tabelas, conceitos e ideias, que identifiquem claramente a fonte original, explicitando as autorizações obtidas dos respectivos proprietários, quando necessárias.\\
3. O autor permite que a UFRJ, por um prazo indeterminado, efetue em qualquer mídia de divulgação, a publicação do trabalho acadêmico em sua totalidade, ou em parte. Essa autorização não envolve ônus de qualquer natureza à UFRJ, ou aos seus representantes.\\
4. O autor pode, excepcionalmente, encaminhar à Comissão de Projeto de Graduação, a não divulgação do material, por um prazo máximo de 01 (um) ano, improrrogável, a contar da data de defesa, desde que o pedido seja justificado, e solicitado antecipadamente, por escrito, à Congregação da Escola Politécnica.\\
5. O autor declara, ainda, ter a capacidade jurídica para a prática do presente ato, assim como ter conhecimento do teor da presente Declaração, estando ciente das sanções e punições legais, no que tange a cópia parcial, ou total, de obra intelectual, o que se configura como violação do direito autoral previsto no Código Penal Brasileiro no art.184 e art.299, bem como na Lei 9.610.\\
6. O autor é o único responsável pelo conteúdo apresentado nos trabalhos acadêmicos publicados, não cabendo à UFRJ, aos seus representantes,  ou ao(s) orientador(es), qualquer responsabilização/ indenização nesse sentido.\\
7. Por ser verdade, firmo a presente declaração.\\

      \vspace{0.4cm}
      \begin{flushright}
         \parbox{10cm}{
            \hrulefill

            \vspace{-.375cm}
            \centering{Bruno Machado Afonso}

            \vspace{0.1cm}
         }
      \end{flushright}
      
\pagebreak

% Copyright
      \vspace{0.5cm}

UNIVERSIDADE FEDERAL DO RIO DE JANEIRO \\
Escola Politécnica - Departamento de Eletrônica e de Computação \\
Centro de Tecnologia, bloco H, sala H-217, Cidade Universitária \\ 
Rio de Janeiro - RJ      CEP 21949-900\\
\vspace{0.5cm}
\paragraph{}Este exemplar é de propriedade da Universidade Federal do Rio de Janeiro, que poderá incluí-lo em base de dados, armazenar em computador, microfilmar ou adotar qualquer forma de arquivamento.
\paragraph{}É permitida a menção, reprodução parcial ou integral e a transmissão entre bibliotecas deste trabalho, sem modificação de seu texto, em qualquer meio que esteja ou venha a ser fixado, para pesquisa acadêmica, comentários e citações, desde que sem finalidade comercial e que seja feita a referência bibliográfica completa.
\paragraph{}Os conceitos expressos neste trabalho são de responsabilidade do(s) autor(es).

\pagebreak


% Agradecimento
\begin{center}
\textbf{AGRADECIMENTO}
\end{center}
      \vspace{0.5cm}

Agradeço, em primeiro lugar, à Deus que guiou todos os meus passos nessa jornada acadêmica e continua a guiar os caminhos da minha vida.
Sem Ele, não seria capaz de terminar este projeto e não chegaria nos patamares onde estou. A Ele a glória.

Gostaria de agradecer também aos meus pais, em especial à minha mãe, que graças ao seu infinito empenho, dedicação, amor e carinho que me impulsionavam
a cada dia, estou hoje concluindo esta etapa de minha vida.

Não menos importante, agradeço também à minha orientadora, Mariane Rembold Petraglia, que dedicou seu tempo para que eu pudesse me tornar um profissional
ainda melhor. Agradeço pela paciência, atenção, boa vontade e o voto de confiança que foram depositados em mim.

Agradeço também aos meus colegas de faculdade, pois sem eles eu não estaria nem no meu último período da graduação para finalizar o meu projeto.
Em especial, agradeço aos meus amigos Felipe Claudio e Diogo Nocera, ambos por serem os mais presentes tanto na jornada acadêmica quanto para 
o resto da minha vida.

\pagebreak

% Resumo
\begin{center}
\textbf{RESUMO}
\end{center}
      \vspace{0.5cm}

O tema de reconhecimento de voz se torna cada vez mais relevante graças ao seu amplo uso tecnológico na sociedade, desde 
assistentes pessoais em \textit{smartphones} e automação residencial, até autenticação por voz para aplicações de segurança.

Uma das características mais importantes neste tema é a detecção da Resposta ao Impulso de Salas (RIR), que representa
o modelo acústico do ambiente. A RIR é usada no processamento de áudio para identificação e reconhecimento de fontes sonoras em campo
distante, que é formada, no caso do tema de reconhecimento de voz, por uma amostra de voz anecóica convoluída com a RIR, acrescida de um ruído,
que pode ser pontual ou disperso.

Um dos desafios no reconhecimento de voz é a estimação da RIR em um sinal de voz em campo distante.
Além das técnicas tradicionais de processamento de sinais, diversas soluções de \textit{deep learning} foram propostas para a estimação da RIR,
contudo estas acabam sendo limitadas devido à falta de variedade e quantidade de bases de RIRs medidas
disponíveis para treinamento de redes neurais.

Neste contexto, o objetivo deste projeto é de desenvolver um algoritmo, usando técnicas de \textit{data augmentation}, que gera amostras de voz
em campo distante (AVCD), construindo assim uma base de dados para uso em treinamentos de soluções de \textit{deep learning}. 
O algoritmo é composto por dois segmentos de \textit{data augmentation}: o primeiro modifica as características de razão direto-reverberante (DRR)
e tempo de reverberação (T60) do sinal acústico, partindo de RIRs reais, gerando RIRs simuladas (RIRSM);
o segundo gera AVCDs, convoluindo amostras de voz anecóicas com as RIRSMs e adicionando ruídos à voz reverberada. 
Ao final do trabalho, são exibidos exemplos de AVCDs geradas pelo algoritmo proposto, analisando se os dados gerados são válidos para uso 
em treinamento de redes neurais.

\paragraph{}
\noindent Palavras-Chave: Resposta ao Impulso de sala, \textit{data augmentation}, \textit{deep learning}, reconhecimento de voz.

\pagebreak


% Abstract
\begin{center}
\textbf{ABSTRACT}
\end{center}
      \vspace{0.5cm}

Speech recognition is a very relevant topic in the present days due to its vast technological usage on modern society
from personal assistants on smartphones and residential automated systems, to voice authentication for security applications.

One of the most important characteristics on this topic is the Room Impulse Response (RIR) detection, which represents the
acoustic model of the room. The RIR is used on signal processing to identify and recognize far-field audio sources,
which for the speech recognition topic, is the anechoic voice sample convolved with the RIR plus noise signal.

One of the challenges when it comes to speech recognition is to estimate the RIR in a far-field voice sample.
Beyond the traditional signal processing algorithms, many deep learning solutions are proposed for the RIR estimation,
however they end up with limited results due to the lack of variety e quantity of RIR databases available for training.

In this context, the main objective of this project is to develop an algorithm using data augmentation techniques that will
generate far-field voice samples, therefore building a database for deep learning training.
The algorithm is composed of two segments: the first modifies the real RIRs characteristics of direct-to-reverberant ratio (DRR) and
the reverberation time (T60) of the acoustic signal generating simulated RIRs (RIRSM);
the second generates far-field voice samples using the previously created RIRSMs, anechoic voice samples and noise signals.
At the end of this work, examples of the generated far-field voice samples by the algorithm are shown and they are analysed to see
if they are valid to be used in neural network training.

\paragraph{}
\noindent Key-words: Room Impulse Response, data augmentation, deep learning, voice recognition.

\pagebreak


% Siglas
\begin{center}
\textbf{SIGLAS}
\end{center}
      \vspace{0.5cm}

\paragraph{}AIR - Aachen Impulse Response
\paragraph{}AVA - Amostra de voz anecóica
\paragraph{}AVCD - Amostra de voz em campo-distante
\paragraph{}AVR - Amostra de voz reverberada
\paragraph{}DA - \textit{Data Augmentation}
\paragraph{}DL - \textit{Deep Learning}
\paragraph{}DRR - Razão Direto-Reverberante
\paragraph{}DTMF - Dual-Tone Multi-Frequency
\paragraph{}RIR - Resposta ao Impulso de Sala
\paragraph{}RIRDA - \textit{Data Augmentation} da Resposta ao Impulso de Sala
\paragraph{}RIRO - Resposta ao Impulso de Sala Original
\paragraph{}RIRSM - Resposta ao Impulso de Sala Simulada
\paragraph{}SNR - Razão Sinal-Ruído
\paragraph{}SRF - Sinal de ruído de fundo
\paragraph{}SRP - Sinal de ruído pontual
\paragraph{}T20 - Tempo de Reverberação (queda de 20 dB)
\paragraph{}T30 - Tempo de Reverberação (queda de 30 dB)
\paragraph{}T60 - Tempo de Reverberação (queda de 60 dB)
\paragraph{}VA - Voz anecóica
\paragraph{}VR - Voz reverberada


\pagebreak







