Neste capítulo, serão introduzidos os principais tópicos do projeto, além de mostrar sua relevância para o escopo da engenharia moderna
e as metodologias que são usadas para alcançar seus objetivos. Ao final é descrita a estrutura organizacional do texto.

\section{Tema}

O tema do trabalho é sobre o estudo de uma forma de simular Respostas ao Impulso de Salas (RIR) com parametrizações diferentes a partir de amostras 
de RIR gravadas em ambientes reais, e ainda usar a RIR para gerar amostras de áudio em locais simulados a partir de gravações de voz reais.

\section{Delimitação}

O estudo é focado em inferir uma técnica de reforço de dados tanto em amostras reais de RIR quanto nas gravações de voz. 
Este trabalho está delimitado em apenas modificar amostras reais de áudio, e não gerar amostras simuladas sem uma gravação de base
, e modificar RIRs medidas, sem a utilização de programas de simulação acústica de salas.

\section{Justificativa}

Com o avanço das tecnologias de automação residencial, assistentes pessoais nos \textit{smartphones} e comunicação \textit{online}, o estudo de técnicas de
processamento de áudio (no caso específico deste trabalho, relacionados à voz), tornou-se mais relevante para a sociedade.
Uma das características mais importantes a ser detectada no processamento de áudio é a Resposta ao Impulso de Salas, 
que representa o modelo acústico do ambiente, pois através desta é possível extrair informações pertinentes do local em que o áudio foi gravado
e também detectar a posição de fontes sonoras e as isolar para reconhecimento.
No âmbito da área de reconhecimento de voz, a fala reverberante, ou seja, o sinal de fala combinado com o modelo acústico do ambiente
é um dos desafios encontrados para a detecção da voz, tornando a identificação da RIR de vital importância para o reconhecimento de fala \cite{FAR-FIELD_ASR}.

Junto a isso, houve avanços no âmbito do aprendizado de máquina, fornecendo alternativas para os métodos tradicionais
de processamento de áudio \cite{ML_Speech_Rec}.
Modelos de arquitetura de redes neurais necessitam de um grande volume de dados para que sejam treinados e aprimorados, e um dos mais recentes
desafios nessa área é o fato das bases de RIR não serem extensas, conforme esclarecidas no artigo \cite{Estimation_RT_DRR},
pois realizar uma grande quantidade de gravações de áudio é uma tarefa de alto custo tanto financeiro quanto de tempo, necessitando de equipamento especializado
e diversos locais com características de modelo sonoro diferentes e pessoas diversas para gerar grande diversidade de amostras de voz.


\section{Objetivos}

O objetivo deste trabalho é desenvolver um algoritmo capaz de gerar amostras de RIR simuladas para diferentes ambientes a partir de uma RIR real e
gerar um banco de dados de amostras de voz convoluídas com as RIR simuladas e com ruídos para uso em treinamento de redes neurais.
Dessa forma, têm-se como objetivos específicos:

\begin{enumerate}
      \item Propor um algoritmo que altere as características da RIR para simular diferentes ambientes com RIR diferentes;
      \item Elaborar um algoritmo que faça o acréscimo de ruídos pontuais ou ruídos de fundo em uma amostra de voz;
      \item Desenvolver um sistema computacional que aplique ambos os algoritmos anteriores em sequência para gerar
      amostras de voz em ambientes ruidosos.
\end{enumerate}


\section{Metodologia}

Um sinal de voz gravado em um ambiente pode ser interpretado como a junção de três partes: uma amostra de voz pura, sem nenhum fator externo
ou reverberação envolvido, convoluída com a RIR onde ocorre a gravação, somada a um sinal de ruído, podendo este 
ser pontual ou um ruído de ambiente. A RIR representa um modelo acústico do ambiente, que define como um receptor acústico irá receber caso o áudio
seja gerado e percebido de dentro deste ambiente. Uma definição de RIR é a de uma função que registra a pressão sonora temporalmente
em um ambiente fechado após uma excitação extremamente curta e cheia de energia (impulso de Dirac).

Neste trabalho é proposta uma forma de gerar RIRs simuladas partindo de uma RIR real, ou seja, obtida de medições da resposta ao impulso em um ambiente
fechado real, e alterando suas propriedades. Reproduz-se o que foi proposto no artigo de \textit{data augmentation} para respostas ao impulso usada na
estimação do modelo acústico \cite{RIR_Data_Aug}, onde são geradas RIRs simuladas, modificando-se as propriedades de Tempo de Decaimento (T60) e de
razão entre áudio direto e reverberado (DRR). Através da alteração dessas duas propriedades,
obtém-se artificialmente uma quantidade representativa de RIRs possíveis de serem gravadas.

Para gerar as amostras de vozes reverberadas que compõe a base de dados, acompanha-se o que é proposto no artigo de estudo de data
augmentation em vozes reverberadas \cite{Speech_Rec}, onde são convoluídos sinais de voz anecóios com as RIRs simuladas que foram geradas anteriormente.
Além disso, é acrescentado a esse sinal de voz reverberado ruídos diversos, que são caracterizados de duas formas: ruídos pontuais e de ambiente.
Os ruídos pontuais são amostras de áudio curtas que podem ser introduzidos em qualquer momento da fala; já os ruídos de ambiente são sons constantemente
presentes ao fundo da gravação para simular um ambiente externo ruidoso. Os ruídos foram extraídos da biblioteca MUSAN \cite{noiseLib}.

Através desses dois passos, são gerados vários sinais de voz reverberados artificialmente. A simulação da RIR tem por objetivo colocar
a amostra de voz em vários ambientes fechados, e a inclusão de ruídos ajudam drasticamente no treinamento de redes neurais, impedindo que as redes fiquem
viciadas em características muito específicas da fala durante o treinamento, uma vez que tendem a simular 
os fatores externos que podem estar envolvidos em uma gravação real.


\section{Descrição}

\paragraph{}O Capítulo 2 apresenta uma breve análise sobre as principais aplicações do tema e os desafios que este trabalho auxilia na solução.

\paragraph{}No Capítulo 3 será descrita a metodologia usada para fazer a \textit{data augmentation} de uma RIR já existente.

\paragraph{}No Capítulo 4 explica-se a metodologia usada para gerar sinais de voz acústicos a partir de RIRs simuladas anteriormente e da adição
de ruídos pontuais ou de fundo.

\paragraph{}O Capítulo 5 apresenta as bases de dados que serão usadas para gerar os resultados experimentais.

\paragraph{}O Capítulo 6 é focado em exibir os resultados obtidos através dos métodos anteriores e demonstrar sua eficácia.

\paragraph{}Por fim, o Capítulo 7 trata das conclusões que podem ser obtidas sobre este projeto, além de propor trabalhos futuros.
