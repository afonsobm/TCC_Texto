\section{Tema}

\paragraph{}Falar do que se trata o trabalho usando uma vis�o macrosc�pica (tamanho do texto: 1 ou 2 par�grafos no m�ximo).

\paragraph{}Sobre que grande �rea de conhecimento voc� vai falar?

\paragraph{}Dada esta grande �rea, qual � o subconjunto de conhecimento sobre o qual ser� o seu trabalho?

\paragraph{}Qual o problema a ser resolvido?

õ

\section{Delimita��o}

\paragraph{}Realizar uma delimita��o informando de quem � a demanda, em que local, e em que momento no tempo. Eventualmente, pode ser mais f�cil come�ar pensando por exclus�o, ou seja, para quem n�o serve, onde n�o deve ser aplicado, e em seguida pegar o universo que sobra (tamanho do texto: livre).


\section{Justificativa}

\paragraph{}Apresentar o porqu� do tema ser interessante de ser estudado. Cuidado, n�o � a motiva��o particular. Devem ser apresentadas raz�es para que algu�m deva se interessar no assunto, e n�o quais foram suas raz�es particulares que motivaram voc� a estud�-lo (tamanho do texto: livre).


\section{Objetivos}

\paragraph{}Informar qual � o objetivo geral do trabalho, isto �, aquilo que deve ser atendido e que corresponde ao indicador inequ�voco do sucesso do seu trabalho. Pode acontecer que venha a existir um conjunto de objetivos espec�ficos, que complementam o objetivo geral (tamanho do texto: livre, mas cuidado para n�o fazer uma literatura romanceada, afinal esta se��o trata dos objetivos).


\section{Metodologia}

\paragraph{}Como � a abordagem do assunto. Como foi feita a pesquisa, se vai houve valida��o, etc. Em resumo, voc� de explicar qual foi sua estrat�gia para atender ao objetivo do trabalho (tamanho do texto: livre).


\section{Descri��o}

\paragraph{}No cap�tulo 2 ser� .....

\paragraph{}O cap�tulo 3 apresenta ...

\paragraph{}Os .... s�o apresentados no cap�tulo 4. Nele ser� explicitado ...

\paragraph{}E assim vai at� chegar na conclus�o.
