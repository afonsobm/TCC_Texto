Este capítulo apresenta as bases de dados que serão usadas para gerar os resultados experimentais.
Para a aplicação da metodologia deste trabalho, é necessário três fontes de dados:

\begin{itemize}
    \item base de dados com amostras de voz anecoicas para convolução as RIRSMs,
    \item base de dados com RIRs reais para o primeiro nível de \textit{Data Augmentation}, gerando as RIRSMs,
    \item base de dados com SRPs e SRFs para o segundo nível de \textit{Data Augmentation}, gerando as AVCDs.
\end{itemize}


\section{Base de amostras de voz anecoicas}

\section{Base de RIRs - Aachen Impulse Response database (AIR)}

A base de AIR \cite{AIR_Database} é um conjunto de respostas ao impulso sonoras que foram medidas em diversas salas.
O objetivo dessa base é de fornecer dados para estudos de algoritmos de processamento de sinais para ambientes reverberados.

Ela é composta primariamente por RIRs binaurais medidas com ou sem uma cabeça falsa de manequim em locais com diferentes
propriedades acústicas; é importante frisar também que a base possui gravações com diferentes distâncias entre a fonte sonora
e os microfones para a mesma sala, gerando outros tempos de reverberação.
A base também possui gravações em diferentes ângulos azimutais com o objetivo de auxiliar algoritmos de detecção
de direção da fonte sonora; para o escopo deste projeto, tais RIRs não serão usadas.

\begin{table} [H]
    \centering
    \caption{Configurações de RIRs disponíveis na AIR.}
    \label{tbl:rir}
    \begin{tabular}{l|c|c|c|l}
        
        \textbf{Sala} & \textbf{Descrição} & \textbf{Canais} & \textbf{Cabeça} & \textbf{Distâncias (m)} \\
        \hline 

        Booth & cabine de estúdio & E/D & S/N & 0,5/1/1,5 \\
        Office & escritório comercial & E/D & S/N & 1/2/3 \\
        Meeting & sala de reuniões & E/D & S/N & 1,45/1,7/1,9/2,25/2,8 \\
        Lecture & sala de aula & E/D & S/N & 2,25/4/5,56/7,1/8,68/10,2 \\
        Stairway & escadaria aberta & E/D & S/N & 1/2/3 \\
        Aula Carolina & igreja de área 570m² & E/D & S/N & 1/2/3/5/15/20 

    \end{tabular}
\end{table}

Os ambientes em que foram feitas as gravações de RIRs e suas respectivas configurações são definidos na tabela \ref{tbl:rir}.
Todos os ambientes usados possuem gravações com ambos os canais esquerdo e direito (E/D) e com configuração com ou sem a cabeça
falsa. As RIRs foram salvas como vetores binários de precisão dupla de ponto flutuante (formato MAT, que pode ser importado
via MATLAB\textregistered).

\section{Base de ruídos - MUSAN}

A base de MUSAN (\textit{A Music, Speech, and Noise Corpus}) \cite{noiseLib} consiste em um conjunto de músicas de diversos gêneros,
amostras de voz de doze línguas e uma variedade de ruídos técnicos e não-técnicos.
Ela foi criada primariamente para auxiliar no treinamento de modelos voltados para detecção de atividade de voz, contudo ela 
é usada também para teste de algoritmos processamento de sinais na área de áudio, por exemplo, reconhecimento de voz e orador.
Uma das vantagens dessa base é o fato dela ser uma compilação de áudios com fontes em domínios públicos, facilitando a 
distribuição dos áudios para uso da comunidade científica.

No escopo deste projeto, será usado somente a seção de ruídos da base, composta por seis horas de áudio no total.
A seção de ruídos é composta por sons técnicos de curta duração, que são usados como SRPs no segundo processo de 
\textit{Data Augmentation}, e por sons de ambiente, usados como SRFs no mesmo processo.

\begin{table} [H]
    \centering
    \caption{Descrição dos tipos de ruídos que constituem parte da base MUSAN.}
    \label{tbl:noise}
    \begin{tabular}{l|c|c|c|l}
        
        \textbf{Sala} & \textbf{Descrição} & \textbf{Canais} & \textbf{Cabeça} & \textbf{Distâncias (m)} \\
        \hline 

        Booth & cabine de estúdio & E/D & S/N & 0,5/1/1,5 \\

    \end{tabular}
\end{table}

A tabela \ref{tbl:noise} indica os ruídos separados da base para gerar AVCDs.
Os arquivos de áudio são disponibilizados em formato WAV, com frequência de amostragem de 16 KHz. 