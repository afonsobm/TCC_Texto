Este capítulo é dedicado à introdução do leitor ao principal tópico de estudo do projeto e assim mostrar
algumas aplicações onde este é usado, além de apresentar os desafios relacionados à estas aplicações.

\section{Resposta ao Impulso de Ambiente Acústico e suas Aplicações}

Dentre os diversos tópicos na grande área de estudo de sinais de áudio, destaca-se a detecção e reconhecimento de fontes acústicas no espaço físico.
Um caso específico deste tópico é sobre sinais de voz gravados em ambientes fechados, onde um ou mais microfones são posicionados no ambiente afastados
da fonte sonora, normalmente uma pessoa que performa a gravação.
Estes sinais são corrompidos pela reverberação do ambiente, que surge a partir da sobreposição da onda sonora anecoica que chega ao microfone com a 
onda sonora atenuada e refletida nas paredes do ambiente fechado.
Este sinal pode ser modelado da seguinte forma:

\begin{equation}
    Y(t) = s(t) \ast h(t) + n(t)
\end{equation}

Onde $Y(t)$ representa o sinal de voz em campo distante, $s(t)$ o sinal de voz anecoico e sem atenuação, $h(t)$ a RIR e $n(t)$ o sinal de ruído que pode
estar presente no ambiente.

% TODO: completar isso
Este trabalho é focado em RIR. 

% SPEECH signals recorded in an enclosed space bymicro-phones placed at a distance from the
%  source are often cor-rupted by reverberation, which arises fromthe superposition ofdelayed and attenuated copies of the 
%  anechoic speech signal. Re-verberation causes signal degradation, typically leading to de-creased speech intelligibility [1],
%   [2]and performance deteriora-tion in speech recognition systems [3]–[5]. Hence, many speechcommunication applications such as teleconferencing 
%   applica-tions, voice-controlled systems, or hearing aids, require effec-tive dereverberation algorithms [4]–[6].


\section{Desafios correlacionados a RIR}